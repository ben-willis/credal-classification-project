\chapter{Introduction}

\section{Classification}

Classification is the problem of identifying which class an object belongs to. Each object can be distinguished by a set of properties know as features and each object belongs to a single class. A classifier is an algorithm which, given previous observations and their classes, can determine which class a new observation belongs to \cite{Theodoridis03}. There are many applications of classifiers, ranging from image recognition to sentiment analysis.

Formally, let us denote the class variable by $C$, taking values in the set $\mathcal{C}$. Also we measure $k$ features $A_1,\dots,A_k$ from the sets $\mathcal{A}_1,\dots,\mathcal{A}_k$. We denote observations of these variables as $c$ and $a_1,\dots,a_k$ respectively.

\section{Auto mobile Insurance}

We will study the problem of classifying the risk to an insurer of a car and comparing this solution to the classification of an expert. We will then examine how both classifications compare to the normalised loss to the insurer. The dataset we're analysing contains 24 attributes that give information about the type of vehicle. These range from number of doors to engine location. The data set also contains a risk assigned by an expert. This risk is initially assigned by price and then, if it is more (or less) risky, shifted up (or down). This is known as symboling \cite{Automobile}.



	
