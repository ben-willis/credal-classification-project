\chapter{Introduction}

We will start by building the classifier beginning with simple maximum likelihood estimates for the chances before introducing a prior probability. We will then introduce the idea of a credal classifier which returns a set of possible classes as oposed to a single class. Finally we will see how this classifier can be adapted to work with missing data. At each stage we will apply our classifier to the problem of determining the risk to an insurer of a car.

\section{Classification}

Classification is the problem of identifying which class an object belongs to. Each object can be distinguished by a set of properties know as features and each object belongs to a single class. A classifier is an algorithm which, given previous observations and their classes, can determine which class a new observation belongs to \cite{Theodoridis03}. There are many applications of classifiers, ranging from image recognition to sentiment analysis.

\section{Auto mobile Insurance}

We will study the problem of classifying the risk to an insurer of a car and comparing this solution to the classification of an expert. We will then examine how both classifications compare to the normalised loss to the insurer.

The data set we will be analysing contains vehicular information about 205 auto mobiles. This features includes dimensions, engine specifications and vehicle characteristics. It also contains an experts assessed risk to the insurer of the vehicle on an integer scale of -2 to 3 with 3 being most risky and -2 being least risky. In addition to the technical information and the experts assessment the data set also contains the normalized loss to the insurer. This ranges from 65 to 256 and is normalized for all vehicles within a particular size classification (two-door small, station wagons, etc) and represents the average loss per car per year \cite{Automobile}.

Initially we will discard objects with missing values and discretize all continuous attributes.



	
