\chapter{Imprecise Prior}

When estimating the probabilities for our classifier we take in to account our prior beliefs for them and the likelihood given a set of observations.

When we initially chose our prior distribution we chose the parameters \cref{initial prior} to represent our indifference towards the classes and attributes.
However this prior is still informative and does not truly represent our lack of prior knowledge about the parameters.
We need a way to represent this lack of knowledge.

The imprecise Dirichlet model is a model for this lack of knowledge.
Instead of using a single prior distribution to represent our beliefs about the unknown parameters we use a set of prior distributions.
In the imprecise Dirichlet models the prior distributions are Dirichlet distribution however we will use a set of the distributions in \cref{prior} for a fixed value of $s$.
The parameter s represents the strength of our prior beliefs and determines how quickly our classifier learns.

Using this model we can find the upper and lower bounds for each posterior expectation based on different prior distributions.
We can then take these to be upper and lower estimates for the probabilities.
For our distributions the upper and lower bounds are given by:
\begin{align}
	\overline{E}(\theta_c \mid s, \mathbf{n}) & = \frac{n(c) + s}{N+s} = \overline{P}(c) \\
	\underline{E}(\theta_c \mid s, \mathbf{n}) & = \frac{n(c)}{N+s} = \underline{P}(c)
\end{align}
These occur when we use the prior distributions with $t(c) = 0$ and $t(c) = 1$ respectively.

Similarly we have:
\begin{align}
	\overline{E}(\theta_{a_i \mid c} \mid s, \mathbf{n}) & = \frac{n(a_i, c) + s}{n(c)+s} = \overline{P}(a_i \mid c) \\
	\underline{E}(\theta_{a_i \mid c} \mid s, \mathbf{n}) & = \frac{n(a_i, c)}{n(c)+s} = \underline{P}(a_i \mid c)
\end{align}
These occur when we use the prior distrPibutions with $t(c) = 1$, $t(a_i, c)=1$ and $t(c) = 1$, $t(a_i, c)=0$ respectively.

Now that we have intervals for these probabilities we need a decision mechanism to assign a risk rating to a vehicle.
Let's start by comparing how our classifier behaves if we assume the true probability is at each end of the interval.
We will use the 0-1 loss function for the actual classification and estimate each probability $P(\cdot)$ to be either the upper or lower probability of our interval.
We will measure accuracy and random classifications as before.

\begin{center}
	\begin{tabular}{l|c c}
	                & Accuracy & Random Assignments \\
	\hline
	Lower Estimates & 62.17\%  & 20.02\%            \\
	Upper Estimates & 10.88\%  & 0\%                \\
	\end{tabular}
\end{center}

Neither of these offer a sufficient classification.
The lower estimates randomly assign classes too often and the upper estimates have a terrible accuracy.

Alternatively we can turn our classifier into a credal classifier.
A credal classifier assigns a set of classes as opposed to a single class to an object.
We will call the set of classes that are returned the credal set.
To use these intervals to create a credal classifier we look at the intervals created by the upper and lower estimates:
\begin{equation}
	\left[ \underline{P}(c)\prod_{i=1}^k \underline{P}(a_i \mid c), \ \overline{P}(c)\prod_{i=1}^k \overline{P}(a_i \mid c) \right]
\end{equation}
For each $c \in \mathcal{C}$.
We say that a class $c$ is dominated if it's interval's upper bound is smaller than another class's interval's lower bound.
For example consider the following intervals:

\begin{center}
	\begin{tabular}{l|c c}
	Risk Rating & Lower Bound & Upper Bound \\
	\hline
	-2          & 0           & 3.12e-9        \\
	-1          & 0           & 8.91e-16       \\
	0           & 3.81e-15    & 2.30e-13       \\
	1           & 2.19e-09    & 2.34e-8        \\
	2           & 1.88e-13    & 6.22e-11       \\
	3           & 0           & 5.82e-17       \\
	\end{tabular}
\end{center}
We can see the classes -1, 0, 2 and 3 are dominated by 1.
However the risk rating of -2 is not dominated by 1 as $3.12e-9 > 2.19e-09$.
Hence our credal classifier returns the set or risk ratings $\{-2, 1\}$.
Note that in this particular example the true risk rating was 1 so our credal classifer was correct to include it in the set of possible classes.

We now need a way to test this classifier.
We cannot use our previous measure of accuracy as this classifier may not return a single class.
Instead there are a few metrics we can use for our diagnostics:
\begin{description}
	\item[A\%] Accuracy of the credal classifier when a single class is returned
	\item[B\%] Percentage of objects for which the true class is in the credal set
	\item[C\%] Percentage of objects for which the credal set is more than one class
\end{description}

We will vary the choice of the hyper parameter s when measuring these three statistics.
\begin{center}
\begin{tabular}{l|c c c}
        & A\%     & B\%     & C\%     \\
\hline
s = 0.5 & 77.50\% & 88.08\% & 57.51\% \\
s = 1   & 77.70\% & 90.67\% & 81.35\% \\
s = 2   & 75.00\% & 94.30\% & 96.89\% \\
s = 5   & -       & 97.92\% & 100\%   \\
\end{tabular}
\end{center}

We see that the accuracy of our classifier when a single class is returned remains almost constant for the different s parameters.
We also note that this accuracy is greater than the accuracy of our corrected naive Bayes classifier on the same data set.
However we notice that varying the s parameter has an effect on the other two metrics.
Increasing the value of s increases the percentage of objects for which the credal set contains more than one class.
It also increases the percentage of objects for which the true class is in the credal set.

This effect can be easily explained.
Increasing the s value increases the upper bound and decreases the lower bound on each of the probabilities being estimated.
Hence increasing the value of s increases the size of the interval and increasing the size of the interval leads to less intervals being dominated and less classes being excluded.