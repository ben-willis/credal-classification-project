\chapter{Corrected NBC with Dirichlet Prior}

We will introduce a prior distribution for our theta variables to improve our classifier. We will also investigate new ways of measuring how successful our classifier is.

\section{Theory}

We return to our likelihood function \cref{likelihood} for our theta variables. We can introduce a prior probability for these parameters and then aim to maximise the posterior.

The Dirichlet distribution is the multinomial extension of the beta distribution for $x_1,\dots,x_k$ where $x_i \in (0,1)$ and $\sum_{i=1}^k x_i = 1$ with probability density function:
\begin{equation} \label{dirichlet_pdf}
	f(x_1,\dots,x_k \mid \alpha_1,\dots,\alpha_k) = \frac{\Gamma(\sum_{i=1}^k\alpha_i)}{\prod_{i=1}^k\Gamma(\alpha_i)} \prod_{i=1}^k x_i^{\alpha_i - 1}
\end{equation}
where $\Gamma$ is the gamma function and $\alpha_i > 0$.

We can rewrite the prior density of our Dirichlet distribution in a similar manner to our likelihood function. By setting $x_i = \theta_{c,\mathbf{a}}$ the prior distributions become:
\begin{equation} \label{prior}
	f(\mathbf{\theta} \mid \mathbf{t}, s) \propto \prod_{x \in \mathcal{C}} \left[ \theta_c^{st(c) - 1} \prod_{i=1}^k \prod_{a_i \in \mathcal{A}_i} \theta_{a_i \mid c}^{st(c, a_i) - 1} \right]
\end{equation}
where $t(\cdot)$ corresponds to $n(\cdot)$. This prior Dirichlet distribution \cite{Zaffalon01} has the following constraints:
\begin{equation}
	\sum_{c \in \mathcal{C}} t(c) = 1
\end{equation}
\begin{equation}
	\sum_{a_i \in \mathcal{A}_i} t(a_i, c) = t(c)
\end{equation}
\begin{equation}
	t(a_i, c) > 0
\end{equation}
For all $(i, a_i, c)$.

When we multiply our likelihood by this prior density get a posterior in the same form.