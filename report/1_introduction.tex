\chapter{AIntroduction}

\section{Classification}

Classification is the problem of identifying which class an object belongs to.
Each object can be distinguished by a set of properties know as features and each object belongs to a single class.
A classifier is an algorithm which, given previous observations and their classes, can determine which class a new observation belongs to \cite{Theodoridis03}.
There are many applications of classifiers including image recognition, sentiment analysis and medical diagnosis.

There are two types of classifiers, supervised and unsupervised.
Unsupervised classifiers infer classes from the data.
Supervised classifiers are constructed from a set of data for which the true classes are known and this is the type of classifier we will be exploring \cite{Michie94}.

\section{Auto mobile Insurance}

Classifiers have many applications in the finance industry ranging from financial trading \cite{Gerlein16} to credit card fraud detection \cite{Pozzolo15}.
We will study the problem of determining the risk to an insurer of a vehicle.
Initially we will learn from an experts classification of risk.
We will then examine how both the expert's and our classifications compare to the actual normalised loss to the insurer of each vehicle.

The data set we will be analysing contains vehicular information from 205 auto mobiles.
Its features include dimensions, engine specifications and vehicle characteristics.
It also contains an expert's assessed risk to the insurer of the vehicle on an integer scale of -2 to 3 with 3 being most risky and -2 being least risky.
In addition to the technical information and the experts assessment, the data set also contains the normalized loss to the insurer.
This ranges from 65 to 256 and is normalized for all vehicles within a particular size classification (two-door small, station wagons, etc.) and represents the average loss per car per year \cite{Automobile}.