\chapter{Conclusion}

In this chapter we will summarise our findings and put them in to context.

\section{Naive Bayes Classifier}

We applied the naive Bayes classifier to the insurance problem.
We saw that it performed poorly in terms of accuracy however we were able to improve the accuracy by introducing a prior distribution for the parametrised probabilities.

When we set $s=0.01$ we achieved an accuracy of 74.61\% which is alright (comparable to other studies?).
This indicates that while the NBC could be used for the purposes of calssifying insurance risk it may not be the most effective method.

\section{Loss Function}

We varied the loss function used to make the decision in the naive Bayes classifier.
We compared the standard 0-1 loss function to the squared difference loss function and the absolute difference loss function.

We saw that for our problem the 0-1 loss functions and the absolute difference loss function returned the same classification.
We also noted that the squared difference loss function may not return an integer classification.
We saw that if we don't round the classification we slightly reduce the mean squared error between it and the true classification.
However if we do round and measure accuracy whilst we still have a smaller mean squared error we have a lower accuracy.

Overall there is little reason to not use the 0-1 loss function for this particular problem problem.

\section{Naive Credal Classifier}

First of all we discussed Walley's imprecise Dirichlet model and how the we can introduce an imprecise prior to our problem.
We created a rudimentary credal classifier by considering upper and lower probabilities derived from this imprecise model and looking at interval dominance.
We then used the imprecise prior to follow Zaffalon's formulation of the naive Credal classifier.
We applied this classifier to our problem with mixed results.
We saw that this classifier is more determinate than our rudimentary classifier however not necessarily as accurate.

Despite this the NCC would still be useful when classifying insurance risk.
For $s=0.1$ when it returns a single class it has an accuracy of 80\%.
The classifier could be used as a preliminary tool and then vehicles it is indeterminate about could be sent on to an expert.

\section{Future Work}

There are a few features of our problem and data set we did not discuss in this report.

Firstly we discretized all continuous variables.
In the future we could look to model these variables with some kind of distribution.

Secondly the naivety assumption, while effective at simplifying the problem, probably is not valid for this data set.
Attributes such as width and height, and city mpg and rural mpg are in fact highly correlated and not conditionally independent.
Future research could study how to incorporate the lack of independence into the naive Credal classifier.