\documentclass[11pt]{report}

\setlength{\parindent}{0em}
\setlength{\parskip}{0.5em}

\usepackage{url}
\usepackage{amsmath}
\usepackage{cleveref}
\usepackage{ntheorem}
\usepackage[a4paper]{geometry}
\usepackage{pgfplots}
\usepackage{graphicx}
\usepackage{caption}
\usepackage{subcaption}
\usepackage{multirow}
\usepackage{tabularx}
\usepackage{bm}

\pgfplotsset{width=10cm,compat=1.9}

\begin{document}

\title{Credal Classification of Automobile Insurance Risk}
\author{Ben Willis}
\maketitle

\section*{Declaration}
This piece of work is a result of my own work except where it forms an assessment
based on group project work. In the case of a group project, the work
has been prepared in collaboration with other members of the group. Material
from the work of others not involved in the project has been acknowledged and
quotations and paraphrases suitably indicated.
\newpage

\begin{abstract}
	This project investigates how credal classification can be applied to the problem of determining automobile insurance risk.
	Automobiles are assigned a risk rating on a scale of -2 to 3 by an expert.
	In this project we start by building a simple naive Bayes classifier to use technical information about a vehicle to assign a risk rating.
	We find that this classifier has a moderate accuracy.
	We will then look at how the choice of loss function can be used to influence the classifier.
	We show that different choices for the loss function allow us to improve different aspects of the naive Bayes classifier.
	Next we describe Zaffalon's naive credal classifier and apply it to our insurance problem.
	We compare our application of the naive credal classifier to the naive Bayes classifier as well as to some previous applications.
	We see that it offers a more cautious alternative which, when dealing with a small number of observations, often fails to isolate a class.
	Finally we show how the naive credal classifier can successfully make use of objects with missing values.
\end{abstract}

\tableofcontents

\chapter{Introduction}

Classifiers have many applications in the finance industry ranging from financial trading \cite{Gerlein16} to credit card fraud detection \cite{Pozzolo15}.
We will study the problem of determining the risk to an insurer of a vehicle.
Initially we will learn from an experts classification of risk.
We will then examine how both the expert's and our classifications compare to the actual normalised loss to the insurer of each vehicle.

Classification is the problem of identifying which class an object belongs to.
Each object can be distinguished by a set of properties know as features and each object belongs to a single class.
A classifier is an algorithm which, given previous observations and their classes, can determine which class a new observation belongs to \cite{Theodoridis03}.
There are many applications of classifiers including image recognition, sentiment analysis and medical diagnosis.

A credal classifier is a special type of classifier.
Instead of returning a single class for a new observation it returns a set of classes.

The data set we will be analysing contains vehicular information from 205 automobiles.
Its features include dimensions, engine specifications and vehicle characteristics.
It also contains an expert's assessed risk to the insurer of the vehicle on an integer scale of -2 to 3 with 3 being most risky and -2 being least risky.
In addition to the technical information and the experts assessment, the data set also contains the normalized loss to the insurer.
This ranges from 65 to 256 and is normalized for all vehicles within a particular size classification (two-door small, station wagons, etc.) and represents the average loss per car per year \cite{Automobile}.
\chapter{Naive Bayes Classifier}

To begin with we will define a simple probabilistic classifier known as the naive Bayes classifier.
This classifier is based on Bayes theorem:
\begin{equation} \label{Bayes Theorem}
	P(A \mid B) = \frac{P(B \mid A)P(A)}{P(B)}
\end{equation}
This result was first described in Thomas Bayes' posthumous work ``An Essay towards solving a Problem in the Doctrine of Chances'' \cite{Bayes63}.
A Bayesian approach to artificial intelligence has been used since the 1960s, especially in a medical setting \cite{Russell03}.
The naive Bayes classifier which we will study in this chapter has it's roots in information retrieval \cite{Lewis98} however has been applied to  many other problems including breast cancer diagnosis \cite{Dumitru09} and text analysis \cite{Nigam98}.


\section{Basic Classifier}

To help explain how the naive Bayes classifier works we will introduce a new data set from a remote sensing study.
The study measured spectral information in the green, red and infrared wavelengths on three separate dates of $523$ different areas of forest in Japan.
In total each observation has nine continuous attributes and falls in to one of four possible classes: Sugi forest, Hinoki forest, Mixed deciduous forest and other non-forest land.
This data set was chosen as it contained a large number of observations and demonstrates a situation where this type of classifier works well.

Formally, let us denote the class variable by $C$, taking values from the set $\mathcal{C} = \{0,1,2,3\}$ for each of the four types of forest.
We measure 9 features $A_1,\dots,A_9$ which we discretize so that they all take values from the set $\mathcal{A} = \{0,\dots,9\}$.
The observations of the variables are denoted by $c$ and $a_1,\dots,a_9$ respectively.

We are interested in the probability of a forest being of type $c$ given sensor readings $\mathbf{a}$ i.e. $P(c \mid \mathbf{a})$.
Using Bayes theorem we can rewrite this as:
\begin{equation} \label{bayes}
	P(c \mid \mathbf{a}) = \frac{P(c)P(\mathbf{a} \mid c)}{P(\mathbf{a})}
\end{equation}

Moreover we can make use of the naivety assumption.
The naivety assumptions states that the attributes are conditionally independent given the class.
In the context of this data set we are assuming the sensor readings are conditionally independent given the type of forest.
This assumption may not necessarily be accurate however it simplifies the problem.
We can therefore write the probability of observing an object with attributes $a_1, \dots\ a_9$ given it is in class $c$ as:
\begin{equation} \label{naivety}
	P(\mathbf{a} \mid c) = \prod_{i=1}^9 P(a_i \mid c)
\end{equation}

Note that $P(\mathbf{a})$ is independent of $c$ and is just a scaling constant.
So by bringing together \cref{bayes,naivety} we can write:
\begin{equation}
	P(c \mid \mathbf{a}) \propto P(c)\prod_{i=1}^{9}P(a_i \mid c)
\end{equation}

To turn this into a classifier we need a way to decide which class a forest falls into based on these probabilities.
We can introduce a \textit{loss function} to allow us to choose the class.
A standard choice is the 0-1 loss function defined as:
\begin{equation}\label{0-1_loss_function}
	L(c, \hat{c}) = 
	\begin{cases}
		0 & \text{if}\ c = \hat{c} \\
		1 & \text{otherwise}
	\end{cases}
\end{equation}
This function assigns a loss of 1 to any wrong classification regardless of the class that is assigned. 
Under this loss function we can write the expected loss as:
\begin{equation}
	E(L) = \sum L(c, \hat{c})P(c \mid \mathbf{a}) = 1 - P(\hat{c} \mid \mathbf{a})
\end{equation}
So to minimize our expected loss we choose:
\begin{equation} \label{map_estimate}
	\hat c = \arg\max_{c \in \mathcal{C}} P(c)\prod_{i=1}^{k}P(a_i \mid c)
\end{equation}
This is known as the maximum a posteriori (MAP) estimate.
We will investigate other options for the loss function later on in this report.

Now that we have our method for making our decision we need to estimate the required probabilities.

To do so we parametrise these probabilities.
We denote the unknown chances of observing an object in class $c$ by $\theta_c$ and the chance of observing an object in class $c$ with attributes $\mathbf{a}$ by $\theta_{\mathbf{a}, c}$.
Similarly we denote the conditional chances of observing an individual attribute $a_i$ and a set of attributes $\mathbf{a}$ given $C=c$ by $\theta_{a_i \mid c}$ and $\theta_{\mathbf{a} \mid c}$ respectively.

Now we have parametrised the probabilities, we wish to estimate we can consider the likelihood function for $\mathbf{\theta}$, the vector whose elements are the chances $\theta_{c}$ and $\theta_{a_i \mid c}$.
Using our data we denote the frequencies of objects in each class $c$ by $n(c)$ and the number of objects in class $c$ with attribute $a_i$ by $n(a_i, c)$.
For example, returning to the forest type data set, the number of observations of class $0$ is $158$ so $n(0) = 158$.
Note that $\sum_{c \in \mathcal{C}}n(c) = N$ and $\sum_{a_i \in \mathcal{A}_i}n(a_i, c) = n(c)$.
We then consider the vector $\mathbf{n}$ which contains these frequencies.

We can derive the likelihood function for these theta chances given the frequencies described above:
\begin{align} \label{likelihood}
	l(\mathbf{\theta} \mid \mathbf{n}) & =  \prod_{c \in \mathcal{C}} \prod_{\mathbf{a} \in \mathbf{\mathcal{A}}} \theta_{\mathbf{a}, c}^{n(\mathbf{a}, c)} \\
	& = \prod_{c \in \mathcal{C}} \prod_{\mathbf{a} \in \mathbf{\mathcal{A}}} \left[ \theta_{c}^{n(\mathbf{a}, c)} \theta_{\mathbf{a} \mid c}^{n(\mathbf{a}, c)} \right] \\
	& = \prod_{c \in \mathcal{C}} \left[ \theta_{c}^{\sum_{\mathbf{a} \in \mathbf{\mathcal{A}}} n(\mathbf{a}, c)} \prod_{\mathbf{a} \in \mathbf{\mathcal{A}}} \prod_{i=1}^k \theta_{a_i \mid c}^{n(\mathbf{a}, c)} \right] \\
	& \propto \prod_{c \in \mathcal{C}} \left[ \theta_c^{n(c)} \prod_{i=1}^k \prod_{a_i \in \mathcal{A}_i} \theta_{a_i \mid c}^{n(a_i, c)} \right]
\end{align}

Now we want to estimate these probabilities.
A simple estimate for these parameters is the maximum likelihood estimate (MLE).
To find the MLE first we take the log likelihood:
\begin{equation}
	L(\mathbf{\theta} \mid \mathbf{n}) \propto \sum_{c \in \mathcal{C}}  n(c)log(\theta_c) + \sum_{c \in \mathcal{C}} \sum_{i=1}^k \sum_{a_i \in \mathcal{A}_i} n(a_i, c) log(\theta_{a_i \mid c}) 
\end{equation}
We want to maximise the likelihood function.
To do so we maximise each part of the log likelihood function which we will do so through the use of Lagrange multipliers.
This is a strategy for finding local maxima and minima of a function subject to constraints.

For the $\theta_c$ parameters we want to maximise:
\begin{equation}
	f(\mathbf{\theta}, \mathbf{n}) = \sum_{c \in \mathcal{C}}  n(c)log(\theta_c)
\end{equation}
under the constraint:
\begin{equation}\label{theta_c constraint}
	g(\mathbf{\theta}, \mathbf{n}) = \sum_{c \in \mathcal{C}}  \theta_c - 1 = 0
\end{equation}
This gives us our Lagrangian:
\begin{equation}
	\mathcal{L}(\mathbf{\theta}, \mathbf{n}, \lambda) = \sum_{c \in \mathcal{C}}  n(c)log(\theta_c) - \lambda(\sum_{c \in \mathcal{C}}  \theta_c - 1)
\end{equation}

Differentiating with respect to $\theta_c$ we have:
\begin{equation}
	\nabla_{\theta_c} \mathcal{L}(\mathbf{\theta}, \mathbf{n}, \lambda) = \frac{n(c)}{\theta_c} - \lambda
\end{equation}

Setting this to zero and using the constraint in \cref{theta_c constraint} we have maximum likelihood estimate:
\begin{equation}
	\hat\theta_c = \frac{n(c)}{N}
\end{equation}
This is just the relative frequency of observations that fall into that class.
Returning to our example data set we have $N=523$ and $n(0)=158$ so $\hat\theta_0 = \frac{158}{523} \approx 0.302$

Similarly for the $\theta_{a_i \mid c}$ parameters we want to maximize:
\begin{equation}
	f(\mathbf{\theta}, \mathbf{n}) = \sum_{c_i \in \mathcal{A}_i} n(a_i, c) log(\theta_{a_i \mid c})
\end{equation}
for each $c \in \mathcal{C}$ and $i \in \{1,\dots,k\}$ under the constraint:
\begin{equation}
	g(\mathbf{\theta}, \mathbf{n}) = \sum_{a_i \in \mathcal{A}_i}  \theta_{a_i \mid c} - 1 = 0
\end{equation}
We again solve this using the Lagrangian to find the maximum likelihood estimate:
\begin{equation}
	\hat\theta_{a_i \mid c} = \frac{n(a_i, c)}{n(c)}
\end{equation}

We can now estimate the required probabilities and hence have our naive Bayes classifier.
First we estimate the probabilities by taking the maximum likelihood estimate for their parametrisation:
\begin{align}
	P(c) & = \frac{n(c)}{N} \\
	P(a_i \mid c) & = \frac{n(a_i, c)}{n(c)}
\end{align}
Then for an object with an unknown class we choose the MAP estimate given in \cref{map_estimate}.

\section{Diagnostics}

To measure how successful our classifier is we will initially use a technique known as \textit{k-fold cross validation}.
In $k$-fold cross validation we split our dataset into $k$ equally sized groups.
Then for each of these group we train the classifier on all the other groups then test it on the excluded group.
We then average all these measurements to return an estimate for the measurement of our classifier.

The choice of $k$ leads to different types of cross validation.
A special case of cross validation is when $k=n$ (the number of observations).
This is knowns as leave-one-out cross validation \cite{Priddy05}.

Initially we will consider two metrics.
The first is accuracy, this is the percentage of correctly classified objects restricted to the case where there is a single MAP estimate.
The seconds is indeterminate classifications, this is percentage of objects with no unique MAP estimate for their class and usually occurs when $P(c \mid \mathbf{a}) = 0$ for all $c \in \mathcal{C}$.

The accuracy of the classifier is 81.77\% and the percentage of unclassified objects is 2.04\% on the forest data set, using 10-fold cross validation.

Previous work carried out by Johnson et al. \cite{Johnson12} on the data set aimed to improve their classification by weighting training data based on geographical distance.
They used SVMs (Support Vector Machines) to classify the forests and recorded an accuracy of 85.9\%.
This indicates that our relatively simple NBC can still achieve a high degree of accuracy.

\section{Application to Automobile Data set}

Now we've seen an application of the NBC to a straightforward data set we want to apply it to our insurance problem.

To make our automobile data appropriate for this method we discretize the continuous variables into $10$ bins with an equal frequency.

Unlike in the trees data set, in this data set we have objects with missing values for attributes.
We have no mechanism for considering these so we must discard these observations for now.
This reduces our data set from 205 observations to 193 observations.

As before we carry out 10-fold cross validation for the two metrics.
The accuracy of our classifier is 59.95\% on the automobile data set.
This is considerably worse than the example forest data set, moreover 22.45\% of the objects were not classified.

Clearly the classifier performs better on the forest type dataset than on the automobile data set.
There are also general failings in our classifier we can fix to improve its accuracy for both data sets.

Firstly our classifier falls down if there are no observations with attribute $a_j$ and class $c$ in our training set.
In these case the maximum likelihood estimate for $\theta_{a_j \mid c}$ is $0$.
This estimate leads to $P(c \mid \mathbf{a}) = 0$ and would rule out assigning any objects with the attribute $a_j$ to class $c$.
This is especially problematic for small sets of data and contributes to the large percentage of indeterminate classifications we saw when applying the classifier to the automobile insurance data set.
We can tackle this by introducing a prior distribution for the theta chances.

One thing that the automobile data set has that the forest type data set doesn't is missing values.
By discarding the objects with missing values we throw away $20$ observations from an already small data set.
If we could make use of these incomplete observation we should be able to improve our classifier.

We can also make use of the structure in the automobile data set's ordered classes.
The accuracy metric does not take into account how close the classification is.
For example if the true class is 2 an assigned class of 1 should be considered better than an assigned class of -2.
The 0-1 loss function also does not take this into account and a better choice of loss function may prove beneficial.


\chapter{Corrected NBC with Dirichlet Prior}

\section{Theory}

We return to our likelihood function \cref{likelihood} for our theta variables.
We can introduce a prior distribution for these parameters and then consider the posterior distribution.

The Dirichlet distribution is the multinomial extension of the beta distribution for $\theta_1,\dots,\theta_k$ where $\theta_i \in (0,1)$ and $\sum_{i=1}^k \theta_i = 1$ with probability density function:
\begin{equation} \label{dirichlet_pdf}
	f(\theta_1,\dots,\theta_k \mid s, t(1),\dots,t(k)) \propto \prod_{i=1}^k \theta_i^{st(i) - 1}
\end{equation}
where $s > 0$ and each $t(i)>0$ such that $\sum_{i=1}^{k}t(i) = 1$.

We introduce a distribution that is similar to our likelihood as our prior density:
\begin{equation} \label{prior}
	f(\mathbf{\theta} \mid \mathbf{t}, s) \propto \prod_{c \in \mathcal{C}} \left[ \theta_c^{st(c) - 1} \prod_{i=1}^k \prod_{a_i \in \mathcal{A}_i} \theta_{a_i \mid c}^{st(c, a_i) - 1} \right]
\end{equation}
This is in the same form as the likelihood however each $n(\cdot)$ is replaces by $st(\cdot) - 1$.
$s > 0$ is a fixed constant and we have the following constraints on $t(\cdot):$
\begin{align}\label{prior_constraints}
	\sum_{c \in \mathcal{C}} t(c) & = 1 \\
	\sum_{a_i \in \mathcal{A}_i} t(a_i, c) & = t(c) && \forall i, c \\
	t(a_i, c) & > 0 && \forall i, a_i, c
\end{align}

This distribution is the conjugate prior for the likelihood function \cref{likelihood}.
When we multiply our likelihood by this prior density we get a posterior in the same family as this prior.
If the prior has hyper parameters $st(\cdot)$ the posterior will have hyper parameters $st(\dot) + n(\cdot)$.

We can now estimate the parameters by taking the posterior expectation e.g.
\begin{equation}
	E(\theta_c|\mathbf{n},s,\mathbf{t})=\hat{\theta_c} = \frac{n(c) + st(c)}{N + s}
\end{equation}

We estimate:
\begin{align}
	P(c) & \text{ by } \frac{n(c) + st(c)}{N + s} \\
	P(a_i \mid c) & \text{ by } \frac{n(a_i, c) + st(a_i, c)}{n(c) + st(c)}
\end{align}

\section{Application}

To apply this classifier to our data set we need to choose the hyper parameters of the prior distribution.
We note that $s$ affects the speed at which our classifier learns and $t(\cdot)$ represents our beliefs for $\theta_\cdot$.
To comply with the constraints \ref{prior_constraints} let us set:
\begin{align}
	s & = 1 \\
	t(c) & = \frac{1}{|C|} \\
	t(a_i, c) & = \frac{1}{|A_i||C|}
\end{align}

Previously we've considered the accuracy of our classifier.
An alternative metric is the mean squared error of our estimate.
\begin{equation}
	\text{MSE} = \frac{1}{n}\sum_{i=1}^n(\hat{c_i} - c_i)^2
\end{equation}

\begin{center}
	\begin{tabular}{ c|c c c c c c }
		              & Accuracy & MSE   & Failed Classifications\\
		\hline
		NBC           & 59.95\%  & 2.823 & 22.45\% \\
		Corrected NBC & 68.17\%  & 0.689 & 0\%
	\end{tabular}
\end{center}

This again shows an improvement over the previous classifier.

\section{Conclusions}
\chapter{Alternate Loss Functions}

In chapter two we introduced the idea of a loss function.
We used the 0-1 loss function to make decision of which class to place an object in given its attributes.
In this chapter we will investigate some alternate loss functions that make better use of the structure of the insurance problem.

When classifying the insurance risk of a vehicle we return a class on an integer of scale of -2 to 3.
The 0-1 loss function assigns a loss of 1 to any risk rating that is not the true risk rating, regardless of how close it is.
We will test alternate choices for the loss function by measuring the accuracy and mean squared error.
We will use the expected posterior estimates with the uniform prior as described in the previous chapter.

\section{The Loss Functions}
\subsection{0-1 Loss Function}
Previously we considered the 0-1 loss function.
To recap this is defined by:
\begin{equation}
	L(c, \hat{c}) = 
	\begin{cases}
		0 & \text{if}\ c = \hat{c} \\
		1 & \text{otherwise}
	\end{cases}
\end{equation}

The expected loss is:
\begin{equation}
	E(L) = \sum_{c \in \mathcal{C}} L(c, \hat{c})P(c \mid \mathbf{a}) = 1 - P(\hat{c} \mid \mathbf{a})
\end{equation}

So to minimize our expected loss we choose:
\begin{equation}
	\hat c = \arg\max_{c \in \mathcal{C}} P(c)\prod_{i=1}^{k}P(a_i \mid c)
\end{equation}
This is known as the maximum a posteriori (MAP) estimate.

\subsection{Squared Differences}
The squared differences loss function is defined as:
\begin{equation}
	L(c, \hat{c}) = (c - \hat{c})^2
\end{equation}
This assigns greater loss to risk ratings that are further away from the true value.

For this function the expected loss is:
\begin{equation}
	E(L) = \sum_{c \in \mathcal{C}} (c - \hat{c})^2P(c \mid \mathbf{a}) 
\end{equation}

Differentiating this with respect to $\hat{c}$ gives:
\begin{equation}
	\frac{\partial}{\partial \hat{c}} E(L) = \sum_{c \in \mathcal{C}} (-2c + 2\hat{c})P(c \mid \mathbf{a}) 
\end{equation}
Setting this equal to zero gives:
\begin{align}
	\sum_{c \in \mathcal{C}} cP(c \mid \mathbf{a}) & = \sum_{c \in \mathcal{C}} \hat{c}P(c \mid \mathbf{a}) \\
	E(c \mid \mathbf{a}) & = \hat{c}
\end{align}
So the estimate which minimizes the loss function is the expected class.
In the context of our problem the estimated class must be an integer, however this expected value may not be.

\subsection{Absolute Difference}
Finally we have the absolute difference loss function:
\begin{equation}
	L(c, \hat{c}) = | c - \hat{c} |
\end{equation}
Once again this assigns greater loss to risk ratings that are further away from the true value.

For this function the expected loss is:
\begin{align}
	E(L) & = \sum_{c \in \mathcal{C}} |c - \hat{c}|P(c \mid \mathbf{a}) \\
	     & = \sum_{c \leq \hat{c}} (\hat{c} - c)P(c \mid \mathbf{a}) - \sum_{c \geq \hat{c}} (\hat{c} - c)P(c \mid \mathbf{a})
\end{align}

Differentiating this with respect to $\hat{c}$ gives:
\begin{equation}
	\frac{\partial}{\partial \hat{c}} E(L) = \sum_{c \leq \hat{c}} P(c \mid \mathbf{a}) - \sum_{c \geq \hat{c}} P(c \mid \mathbf{a})
\end{equation}
Setting this equal to zero gives:
\begin{align}
	\sum_{c \leq \hat{c}} P(c \mid \mathbf{a}) & = \sum_{c \geq \hat{c}} P(c \mid \mathbf{a}) \\
	P(c \leq \hat{c} \mid \mathbf{a}) & = P(c \geq \hat{c} \mid \mathbf{a})
\end{align}
So the estimate that minimizes expected loss for this loss function is the median value.
This may be difficult to define on our data set.
For example suppose class -3 had $P(C = -2 \mid \mathbf{a}) = 0.5 = P(C=3 \mid \mathbf{a})$ then any class could reasonably be considered the median and therefore minimize our expected loss.
When more than one risk rating could be reasonably considered the median we will choose one at random.

\section{Application}
We will now apply these loss functions to our auto mobile data set and measure accuracy and mean squared error.
We will also investigate how often the assigned classes agree.

Once again we discretize and discard vehicles with missing attributes.

As the expected posterior estimates perform better than the maximum likelihood estimates we shall use these to estimate the required probabilities.
We will also use the uniform hyper parameters and set $s=1$.

Using 10-fold cross validation our various loss functions perform as follow.

\begin{center}
\begin{tabular}{l|c c}
	Loss Function       & Accuracy & MSE   \\
	\hline
	0-1                 & 68.59\%  & 0.642 \\
	Squared Difference  & 67.48\%  & 0.594 \\
	Absolute Difference & 68.34\%  & 0.635 \\
\end{tabular}
\end{center}

Note the similarity between the 0-1 loss function and the absolute difference loss function.
The squared difference loss function performs slightly worse in the accuracy metric however scores better for MSE.

We can also look at the percentage of time each loss function agrees:
\begin{center}
\begin{tabular}{l|c c c}
	Loss Function       & 0-1     & Squared Difference & Absolute Difference   \\
	\hline
	0-1                 & 100\% & 95.85\% & 99.48\% \\
	Squared Difference  & - & 100\% & 96.37\%\\
	Absolute Difference & - & - & 100\% \\
\end{tabular}
\end{center}

This confirms that the 0-1 loss function and absolute differences loss function assign classes in a very similar manner.
This is due to their often being a $\hat{c} \in \mathcal{C}$ such that $P(\hat{c} \mid \mathbf{a})$ that is much greater than for other choices of $c$ other.
When this is the case $\hat{c}$ is the choice for the 0-1 loss function.
Additionaly if it is greater than 0.5, which is often the case, it is the only choice for the median of the data set and hence is the absolute difference loss functions choice.

The squared difference loss function differs from these two slightly as it is more skewed by outliers.
Consider the case $P(C=-2\mid \mathbf{a})=0.6, P(C=3\mid \mathbf{a})=0.4$ then the expectation for $C$ is $0$ while both the 0-1 and absolute loss functions will choose -2.

\newcommand{\sn}[2]{\ensuremath{{#1}\times 10^{#2}}}

\chapter{Imprecise Prior}

When estimating the probabilities for our classifier we take into account our prior beliefs for them and the likelihood given a set of observations.
When choosing our prior distribution we had to pick hyperparameters without any knowledge.
In this chapter we will use Walley's imprecise Dirichlet model \cite{Walley96} to model this lack of knowledge and then create a simple interval based credal classifier.

\section{Imprecise Dirichlet Model}

When we initially chose our prior distribution we chose the hyperparameters in \cref{initial prior} such that they follow the principle of indifference.
However as previously mentioned this prior does not truly represent a lack of prior knowledge.
We need a way to represent this lack of knowledge and acknowledge that our estimates for the probabilities depend on our choice of hyperparameters.

The imprecise Dirichlet model is a model for this lack of knowledge introduced by Walley \cite{Walley96}.
Instead of using a single prior distribution to represent our beliefs about the unknown parameters we use a set of prior distributions.

In the imprecise Dirichlet model the prior distributions are Dirichlet distributions with parameters $(s, \mathbf{t})$ such that $\sum_{c \in \mathcal{C}} t(c) = 1$.
The distributions for the likelihood are multinomial so:
\begin{equation}
	f(\mathbf{n} \mid \bm{\theta}) \propto \prod_{c \in \mathcal{C}} \theta_c^{n(c)}
	\qquad
	f(\bm{\theta} \mid s, \mathbf{t}) \propto \prod_{c \in \mathcal{C}} \theta_c^{st(c) - 1}
\end{equation}
Then the posterior distribution is of the form:
\begin{equation} \label{dirichlet_pdf2}
	f(\mathbf{\theta} \mid \mathbf{n}, s, \mathbf{t}) \propto \prod_{c \in \mathcal{C}} \theta_c^{n(c) + st(c) - 1}
\end{equation}
which is also a Dirichlet distribution with parameters $(N+s, \frac{\mathbf{n}+s\mathbf{t}}{N+s})$.
We can then take the posterior expectation for the $\bm{\theta}$ chances giving:
\begin{equation}
	P_t(c) = E(\theta_c \mid \mathbf{n}, s, \mathbf{t}) = \frac{n(c)+st(c)}{N+s}
\end{equation}
Note this is a function of $t(c)$ which allow us to obtain upper and lower estimates for for probabilities by varying $t(c)$:
\begin{align}
	\overline{P}(c) & = \frac{n(c)+s}{N+s} & (t(c) \rightarrow 1) \\
	\underline{P}(c) & = \frac{n(c)}{N+s}  & (t(c) \rightarrow 0)
\end{align}

Zaffalon applied this method of imprecise priors to the problem of classification.
We will use a set of the prior distributions of the form in \cref{prior} for a fixed value of $s$.
The parameter $s$ represents the strength of our prior beliefs and determines how quickly our classifier learns.

\section{Imprecise Probabilities}

Imprecise probability refers to the partial specification of a probability for example through upper and lower bounds for a probability.
Imprecise probabilities have other applications in artificial intelligence and can better represent an experts knowledge \cite{Coolen11}.
Using the model of imprecise priors we can find the upper and lower bounds for each posterior expectation based on different prior distributions.

Recall that we estimate the probabilities by taking the expectation of the posterior distribution and that these estimates depend on $\mathbf{t}$ i.e.:
\begin{align}
	P_t(c) & = E(\theta_c \mid \mathbf{n},s,\mathbf{t}) = \frac{n(c) + st(c)}{N + s} \\
	P_t(a_i \mid c) & = E(\theta_{a_i \mid c} \mid \mathbf{n},s,\mathbf{t}) = \frac{n(a_i, c) + st(a_i, c)}{n(c) + st(c)}
\end{align}
We can then use these to find upper and lower estimates for the probabilities over all values of $\mathbf{t}$ in our prior model.
For our distributions the upper and lower bounds are given by:
\begin{align}
	\overline{P}(c) & = \frac{n(c) + s}{N+s} \\
	\underline{P}(c) & = \frac{n(c)}{N+s}
\end{align}
These occur when we use the prior distributions with $t(c) \rightarrow 0$ and $t(c) \rightarrow 1$ respectively.

Similarly we have:
\begin{align}
	\overline{P}(a_i \mid c) & = \frac{n(a_i, c) + s}{n(c)+s} \\
	\underline{P}(a_i \mid c) & = \frac{n(a_i, c)}{n(c)+s}
\end{align}
These occur when we use the prior distributions with $t(c) \rightarrow 1$, $t(a_i, c)\rightarrow1$ and $t(c) \rightarrow 1$, $t(a_i, c)\rightarrow0$ respectively.

Let's start by comparing how our classifier behaves if we assume the true probability is at each end of the interval.
We will use the 0-1 loss function for the decision method and estimate each probability $P(\cdot)$ by either the upper or lower probability of our interval.
We will measure accuracy and indeterminate classifications as before.

\begin{center}
	\begin{tabular}{l|c c}
	                & Accuracy & Indeterminate Assignments \\
	\hline
	Lower Estimates & 62.17\%  & 20.02\%            \\
	Upper Estimates & 40.09\%  & 0\%                \\
	\end{tabular}
\end{center}

Neither of these offer a sufficient classification.
There are often no observations of an attribute with a particular class which is why using the lower estimates leads to indeterminate classifications despite a reasonable accuracy for this data set.
On the other hand when using the upper estimates our classifier has very low accuracy.

\section{Simple Credal Classifier}
Alternatively we can turn our classifier into a credal classifier.
A credal classifier assigns a set of classes as opposed to a single class to an object.

Recall the MAP estimate for the risk rating of a vehicle given by \cref{map}.
We say that a class $c'$ is dominated by $c''$ if:
\begin{equation}\label{Credal Dominance}
	P_t(c' \mid \mathbf{a}) < P_t(c'' \mid \mathbf{a})
\end{equation}
for all values of $\mathbf{t}$ in our prior model.
This is because the MAP estimate for the class will always choose $c''$ over $c'$ regardless of which prior is used.
Note that this is equivalent to $P_t(c', \mathbf{a}) < P_t(c'', \mathbf{a})$ as $P(\mathbf{a})$ does not depend on $c$.

We can use the bounds for the probabilities we found earlier to create an interval $P(c, \mathbf{a})$ must lie in. 

If we look at the intervals created by the upper and lower estimates we achieve:
\begin{equation}
	P_t(c, \mathbf{a}) \in \left[ \underline{P}(c)\prod_{i=1}^k \underline{P}(a_i \mid c), \overline{P}(c)\prod_{i=1}^k \overline{P}(a_i \mid c) \right]
\end{equation}
for each $c \in \mathcal{C}$.
This is true because:
\begin{equation}
\underline{P}(c)\prod_{i=1}^k \underline{P}(a_i \mid c) \leq \underline{P}(c, \mathbf{a}) \leq P_t(c, \mathbf{a}) \leq \overline{P}(c, \mathbf{a}) \leq \overline{P}(c)\prod_{i=1}^k \overline{P}(a_i \mid c)
\end{equation}
for all choices of $\mathbf{t}$.
We can use the above intervals to create a simple credal classifier.

For an example consider the following intervals:
\begin{center}
	\begin{tabular}{l|c c}
	Risk Rating & Lower Bound & Upper Bound \\
	\hline
	-2          & $0$              & $\sn{3.12}{-9}$  \\
	-1          & $0$              & $\sn{8.91}{-16}$ \\
	0           & $\sn{3.81}{-15}$ & $\sn{2.30}{-13}$ \\
	1           & $\sn{2.19}{-9}$  & $\sn{2.34}{-8}$  \\
	2           & $\sn{1.88}{-13}$ & $\sn{6.22}{-11}$ \\
	3           & $0$              & $\sn{5.82}{-17}$ \\
	\end{tabular}
\end{center}
We can see the classes -1, 0, 2 and 3 are dominated by 1.
However the risk rating of -2 is not dominated by 1 as $\sn{3.12}{-9} > \sn{2.19}{-9}$.
Hence our credal classifier returns the set of risk ratings $\{-2, 1\}$.
Note that in this particular example the true risk rating was 1 so our credal classifer was correct to include it in the set of possible classes.

\section{Diagnostics}

We now need a way to test this classifier.
We cannot use our previous measure of accuracy as this classifier may not return a single class.
Instead there are a few metrics we can use for our diagnostics \cite{Antonucci11}:
\begin{description}
	\item[Single Accuracy (A\%)] Accuracy of the credal classifier when a single class is returned
	\item[Set Accuracy (B\%)] Percentage of objects for which the true class is in the returned set when the set has size larger than one
	\item[Indeterminate Output Size (C)] Average set size for returned sets containing more than one class
	\item[Determinacy (D\%)] Percentage of objects for which the returned set contains one class
\end{description}

\section{Application}

We will vary the choice of the hyperparameter $s$ when measuring these three statistics to see its effect.

%Seed = 0.1

\begin{center}
\begin{tabular}{l|c c c c}
        & A\%     & B\%     & C    & D\%     \\
\hline
s = 0.5 & 77.02\% & 87.05\% & 3.71 & 38.34\% \\
s = 1   & 76.92\% & 90.67\% & 3.75 & 20.20\% \\
s = 2   & 75.00\% & 94.30\% & 4.04 & 3.11\% \\
s = 5   & -       & 97.92\% & 5.22 & 0\%   \\
\end{tabular}
\end{center}

We see that the single accuracy of our classifier slightly decreases for the different $s$ parameters.
We also note that this single accuracy is greater than the accuracy of our corrected naive Bayes classifier on the same data set.
However we notice that varying the $s$ parameter has an effect on the other two metrics.
Increasing the value of $s$ decreases determinacy and increases the set accuracy.

This effect can be easily explained.
Increasing the $s$ value increases the upper bound and decreases the lower bound on each of the probabilities being estimated.
Hence increasing the value of $s$ increases the size of the interval and increasing the size of the interval leads to less intervals being dominated and fewer classes being excluded.

\section{Conclusion}

In this chapter we have seen how Walley's imprecise Dirichlet model can be used to create intervals for the probabilities we wish to estimate when using the naive Bayes classifier.
We have seen that using either the upper and lower bounds lead to a very poor classifier.
We then used these imprecise probabilities to create a simple credal classifier.
\chapter{The Naive Credal Classifier}

In the simple credal classifier we estimated the lower and upper bounds for each probability separately using our imprecise prior model.
We then used these separate estimates to make inferences about the true probability of interest: $P(c \mid \mathbf{a})$.

However an alternate method for credal classification was proposed by Zaffalon \cite{Zaffalon01} which we will outline in this section.
This classifier is known as the naive Credal classifier (NCC) and can be more determinate than the simple credal classifier we studied earlier.

\section{Previous Work}

Zaffalon has provided multiple examples of this classifier in action.

In one such study he applies the naive Credal classifier to dementia diagnosis \cite{Zaffalon03}.
He starts with a data set containing test results for 3385 different patients split in to five different categories (four describing types of dementia suffers, one describing normal patients).
The data set also contains missing values for some of the patient's test results.
In the first part of the study he simple distinguishes between dementia sufferers and non-dementia sufferers.
In this part the NCC is able to isolate a single class about 90\% of the time and in these cases is accuracy 95\% of the time.
When it fails to isolate a single class the NBC is only able to classify the same object correctly 70\% of the time.
In the second part he limits the study to the type of dementia thus reducing the data set  1103 observations.
Here we see similarly positive results; when the NCC isolates a single class it is accurate 94\% of the time and when it outputs more than one class the set size is about 2 on average and the true class is in this set 98\% of the time.
This study gives an example when a large data set can lead to the NCC being highly determinate and accurate.

In another study he applies the NCC to environmental mining data \cite{Zaffalon02}.
Here the data falls in to one of four categories however the data set only consists of 155 complete instances which is much closer in size to our insurance problem.
In this study the NCC only produced a single class 60\% of the time and of these classes had a single accuracy of 52\%.
When returning more than one class the true class was contained in the output set 82\% of the time.
In comparison the NBC had an accuracy of 48\% on the whole data set and 43\% on the subset of objects the NCC was indeterminate about.
This provides a good example of a situation where the NCC will withhold judgement due to a lack of information.

\section{Theory}

To define the naive Credal classifier we first rewrite our original definition of credal dominance \cref{Credal Dominance} as:
\begin{equation}
	\frac{P_t(c' \mid \mathbf{a})}{P_t(c'' \mid \mathbf{a})} = \frac{P_t(c')\prod_{i=1}^{k}P_t(a_i \mid c')}{P_t(c'')\prod_{i=1}^{k}P_t(a_i \mid c'')} > 1
\end{equation}
Note that the equality holds because the constant $P(\mathbf{a})$ is cancelled.

If we plug in the posterior expectation for our parametrisation as an estimate for the probabilities then we arrive at:
\begin{equation}
	\frac{n(c')+st(c')}{n(c'')+st(c'')} \prod_{i=1}^k \frac{n(a_i, c') + st(a_i , c')}{n(c'') + st(c'')} \frac{n(c'') + st(c'')}{n(a_i, c') + st(a_i , c')} > 1
\end{equation}

To determine whether this inequality holds we can solve the optimization problem:
\begin{align}
	\min & \left[ \frac{n(c'')+st(c'')}{n(c')+st(c')} \right]^{k-1} \prod_{i=1}^k \frac{n(a_i, c') + st(a_i , c')}{n(a_i, c'') + st(a_i , c'')} \\
	\text{s.t.} & \sum_{c \in \mathcal{C}} t(c) = 1 \\
	& 0 < t(a_i, c) < t(c)
\end{align}
Then compare the answer to 1.
This is the same optimization problem as described by Zaffalon \cite{Zaffalon01}.

It is possible to manipulate this problem into an easier to solve form.
Firstly note that the minimum is achieved when each $t(a_i, c') \rightarrow 0$ and $t(a_i, c'') \rightarrow t(c'')$ so we can use these values in the objective function.
Furthermore, at the minimum, we have $t(c') = 1 - t(c'')$.
To simplify the problem set $st(c'') = x$ then our optimization problem becomes a problem in a single variable:
\begin{align} \label{Credal Dominance Test}
	\min \quad & f(x) = \left[ \frac{n(c'') + x}{n(c') + s - x} \right]^{k-1} \prod_{i=1}^k \frac{n(a_i, c')}{n(a_i, c'') + x} \\
	\text{s.t.} \quad & 0 < x < s
\end{align}

Before we solve this optimization problem we can rule out an edge case.
If $n(a_i, c')=0$ for any $a_i$ then the $c'$ does not dominate $c''$.
IF $n(a_i, c'')=0$ for any $a_i$ then we set $f(0)=10$ to indicate domination is achieved at this point.

Next step is to figure out what the objective function looks like.
Note that it's always positive so if we take the log of $f$ and differentiate we get:
\begin{equation}
	\frac{d\ln(f)}{dx} = \frac{k-1}{n(c'')+x} + \frac{k-1}{n(c')+1-x} - \sum_{i-1}^k \frac{1}{n(a_i, c'') + x}
\end{equation}
Differentiating again gives:
\begin{equation}
	\frac{d^2\ln(f)}{dx^2} = -\frac{k-1}{(n(c'') + x)^2} + \frac{k-1}{(n(c')+1-x)^2} + \sum_{i=1}^k \frac{1}{(n(a_i, c'') + x)^2}
\end{equation}
This is always positive hence the objective function is log concave.
As the logarithm is monotone it follows that $f(x)$ is also concave and hence has a single minimum.

If we remove the edge cases described above we are left with the simple problem of finding the maximum of a concave function in a given interval.
To do so we use the fminbound function in the SciPy library which uses Brent's method \cite{fminbound}.

% If we remove the edge case described above we can compute $f(x)$ for any $x \in [0, s]$.
% There are three cases for finding the minimum:
% \begin{itemize}
% 	\item[$\frac{d\ln f(0)}{dx} \geq 0$] then the minimum is achieved for $x<0$ and hence the minimum in this interval is $f(0)$
% 	\item[$\frac{d\ln f(s)}{dx} \leq 0$] then the minimum is achieved for $x>s$ and hence the minimum in this interval is $f(s)$
% 	\item[Otherwise] the minimum is located within this interval and we can approximate it numerically
% \end{itemize}

\section{Application}

We will measure the same metrics as previously for this new classifier.
The results from 10-fold cross validation with varying $s$ values are as follow:

\begin{tikzpicture}
\begin{axis}[
    xlabel={s},
    ylabel={Percentage \%},
    xmin=0, xmax=5.5,
    ymin=0, ymax=100,
	legend pos=outer north east
]

\addplot[
    color=blue,
    mark=square,
    ]
    coordinates {
    (0.1,79.51)(0.5,77.03)(1,75.56)(2,76.92)
    };
    \label{sng_acc}

\addplot[
    color=red,
    mark=square,
    ]
    coordinates {
    (0.1,86.52)(0.5,88.60)(1,88.60)(2,90.16)(3, 91.19)(4, 92.23)(5,93.26)
    };
    \label{set_acc}
 
\addplot[
    color=green,
    mark=square,
    ]
    coordinates {
    (0.1,63.21)(0.5,38.32)(1,23.31)(2,6.74)(3, 0)(4, 0)(5,0)
    };
    \label{det}

\addlegendimage{/pgfplots/refstyle=sng_acc}\addlegendentry{Single Accuracy}
\addlegendimage{/pgfplots/refstyle=set_acc}\addlegendentry{Set Accuracy}
\addlegendimage{/pgfplots/refstyle=det}\addlegendentry{Determinacy}
\end{axis}

\end{tikzpicture}

Additionally we have the following measurements for indeterminate output size:
\begin{center}
\begin{tabular}{c|c c c c}
s & 0.5 & 1 & 2 & 5 \\
\hline
Indeterminate Output Size & 3.65 & 3.66 & 3.74 & 4.25
\end{tabular}
\end{center}

First we note the increase in indeterminate output size and decrease in determinacy.
These are due to the domination criteria becoming harder to satisfy for larger $s$ values.
This leads to less classes being creedal dominated and larger output size.

This can also explain the slight trends in single and set accuracy.
Set accuracy increases because the indeterminate output size is always increasing so for each increase in $s$ value we are more likely to see the true class added to the oupur set if it was not already there.
On the other hand the single accuracy does not change much.
As we increase the $s$ value we decrease the number of single outputs and it would appear the outputs that become indeterminate are equally likely to be correct classifications as incorrect classifications.

We can also directly compare the classifications of the naive Credal classifier to those of our interval based classifier. For $s=1$ we have:
\begin{center}
\begin{tabular}{l|c c c c}
         & A\%     & B\%     & C    & D\%     \\
\hline
Interval & 75.61\% & 89.12\% & 3.71 & 21.32\% \\
NCC      & 75.56\% & 88.60\% & 3.66 & 23.31\% \\
\end{tabular}
\end{center}
Here we see similar set and single accuracies.
However we see that the naive Credal classifier is more determinate than the interval based classifier and, when indeterminate, has a smaller average output size.

Our data set only contains observations of three vehicles with risk rating -2.
This means that our credal classifiers struggle to eliminate this classification as an option.
When we consider situations where the NCC is indeterminate and returns two possible classes -2 is always one of these classes.
Additionally the other class is the correct classification 88.60\% of the time.
This is a good example of a situation where the NCC reserves judgement due to lack of observations.

\section{Comparison to NBC}

In addition to our interval based credal classifier we can also compare the NCC to the NBC.
We will do this using the same statistics as \cite{Zaffalon01}.
Zaffalon compared the single accuracy of the NCC to the accuracy of the NBC.
He also considered the accuracy of the NBC limited to objects the NCC was indeterminate about.
We will set $s=1$ and consider the same three metrics: NCC single accuracy (A\textsubscript{NCC}\%), NBC accuracy A\textsubscript{NBC} and NBC accuracy on  the NCC's indeterminate objects (A\textsubscript{S}\%).
\begin{center}
\begin{tabular}{c c c}
\hline
A\textsubscript{NCC}\% & A\textsubscript{NBC} & A\textsubscript{s}\% \\
\hline
81.40\%                & 69.43\%              & 66.00\% \\
\hline
\end{tabular}
\end{center}

This demonstrates how the indecision of the naive Credal classifier can hold it back.
We see that when it returns a single class it is far more accurate than the naive Bayes classifier.
However we also see that for objects the NCC is indeterminate about the NBC is only slightly less accurate.
These measurements are in contrast to those achieved by Zaffalon.
When he carried out his analysis he saw that the NBC drop off significantly when only considering the classes the NCC was indeterminate about.

The reason for this difference again comes back to the lack of observations of vehicles with a -2 risk rating.
A lot of the time when the NCC is indeterminate it is only with regards to the true class and the -2 risk rating.
The NBC is more decisive and will opt for the correct class in many of these occasions.

\section{Conclusions}

We've seen that Zaffalon's naive Credal classifier is slightly more determinate than our simple interval based classifier.
We've also seen how our choice of the $s$ hyper parameter affects how cautious our classifier is.

However we've also seen that the naive Credal classifier can be very cautious when there is a small number of observations of a particular class.
We saw that when this is the case it would often return sets of size two containing the true class and the class with a low number of observations.
We saw that this meant the NBC still had a good accuracy when applied to the sets the NCC was indeterminate about because it is a less cautious classifier.
\chapter{Missing Values}

Missing data is a common problem when working with real data.
Previously we had no method for classifying or using the vehicles with missing attributes.
In this chapter we will discuss methods to overcome this issue.
We will then apply two different methods too our problem and see whether they improve our classifier.
Note that in our data set the class is never missing from a vehicle so we only have to deal with missing technical attributes.

\section{Common approaches to missing data}

Before we look at how to deal with missing data we should consider what type of missing According to Rubin \cite{Rubin76} there are three types of missing data:
\begin{description}
	\item[MCAR] The data is missing completely at random if the missing attributes are independent of all other  attributes
	\item[MAR] The data is missing at random if the missing data can be accounted for by other attributes where there is complete information e.g. cars with two doors might be missing engine specifications but this has nothing to do with the engine specifications themselves
	\item[MNAR] The data is missing not at random if the attribute is missing due to value of the attribute e.g. vehicles with a high price are missing their price
\end{description}
Note that there is no way of distinguishing between MAR and MNAR without finding out some of the missing values.

So far we have been removing the vehicles with missing attributes from our data set otherwise known as Listwise Deletion.
Listwise deletion only introduces no bias if the data is MCAR \cite{Allison02} and can be a waste of important information if the number of objects with missing objects is large \cite{Little92}.
Despite these issues it remains the easiest method to implement which is why we have used it thus far.

Most other methods involve replacing the missing values with estimated values and fall under the category of \textit{imputation}.
We will briefly describe a few of the most popular as described by Baraldi and Enders \cite{Baraldi09}.
We could replace the missing values with the mean of the remaining values however this increases correlation between attributes and if the data is not MCAR also biases the mean.
Alternatively regression can be used to impute values which has the benefit of not introducing bias to the means if the data is MCAR or MAR.
However regression imputation also exaggerates the correlation.
Finally we can model the attribute with missing values using a distribution and draw multiple samples to create multiple plausible data sets.
This is known as multiple imputation and was introduced by Rubin \cite{Rubin87}.

In the following two sections we will apply two different methods to our insurance problem.
When using these methods we make no assumptions about the type of missing data involved.

\section{Reduced-feature Models}

One approach to missing data is to apply a different model that only uses the attributes without missing values. This model can be different for each test instance by dropping only the particular columns with missing values for that instance \cite{Saar-Tsechansky07}.
In this section we will use an approach where we remove all attributes from our data set that have missing values.

The method of feature reduction has other applications when working with the naive Bayes classifier.
In one study \cite{Novakovic10} various techniques for removing features that hold less information about the class were trialled with the aim of improving naive Bayes classifier.
The feature reduction techniques had mixed with results however use of one particular technique was found to consistently maintain or improve the classifier.
This indicates that the method of feature reduction does not necessarily lead to a weaker classifier.

For our data set we will remove the attributes with missing values (Price, Number of doors, Bore, Stroke, Horsepower, Peak RPM) from our classifier.
In practice this means we reduce the number of available attributes from 24 to 18.

We can then apply the NCC for $s=1$ in a similar manner:
\begin{center}
\begin{tabular}{l|c c c c}
	                        &   A\%   &   B\%   &  C   &   D\%   \\
	\hline
	Without missing values  & 73.68\% & 88.60\% & 3.63 & 19.69\% \\
	Without missing columns & 74.51\% & 82.44\% & 3.33 & 24.88\% \\
\end{tabular}
\end{center}

We see a small rise in single accuracy but a large decrease in set accuracy.
We also observe a large increase in determinacy.
Removing the features with missing values is having two main effects on our classifier.
Firstly it would appear the NCC is able to use the extra vehicles we are now considering to eliminate more classes.
However the decrease in attributes is also causing the classifier to be less accurate and leading to our classifier dropping the correct class more often.

\section{NCC Approach}

Another method builds upon our test for credal dominance in \cref{Credal Dominance}.
This method was introduced by Zaffalon when he originally introduced the naive Credal classifier \cite{Zaffalon01}.

We return to the optimization problem seen in section 5 \cref{Credal Dominance Test}.
We can factor in our the uncertainty of the frequencies caused by missing values by giving lower and upper bounds for them.
These lower and upper bounds ($\underline{n}(a_i \mid c)$ and $\overline{n}(a_i \mid c)$) are found by considering all possibilities for the missing attributes in our data set.
The optimization problem then becomes:
\begin{align}
	\min \quad & f(x) = \left[ \frac{n(c'') + x}{n(c') + 1 - x} \right]^{k-1} \prod_{i=1}^k \frac{\underline{n}(a_i, c')}{\overline{n}(a_i, c'') + x} \\
	\text{s.t.} \quad & 0 < x < s
\end{align}
We can find the solution to this problem as before.

We will apply this method to our data set and compare it to previous results.
Unlike before we will not vary the $s$ value and instead use $s=1$ throughout.
As before we will measure four metrics: Single Accuracy (A\%), Set Accuracy (B\%), Indeterminate Output Size (C) and Determinacy (D\%).

\begin{center}
\begin{tabular}{l|c c c c}
	                       & A\%     & B\%     & C    & D\%     \\
	\hline
	Without missing values & 73.68\% & 88.60\% & 3.63 & 19.69\% \\
	With missing values    & 78.57\% & 89.12\% & 3.66 & 21.76\% \\
\end{tabular}
\end{center}

When using vehicles with missing attributes we see a large increase in single accuracy and a small increase in set accuracy. We also see an increase in determinacy and indeterminate set size remaining roughly the same.
This indicates that, despite the added uncertainty, the use of vehicle with missing attributes improves our classifier.

\section{Conclusions}

Removing the attributes with missing values is a simple approach for dealing with missing values in our data set.
However for our data set this significantly reduces the set accuracy of our classifier.
Furthermore this approach would not be appropriate if we had more attributes with missing values.

However using the uncertainty about frequencies in the formulation of the naive Credal classifier improves the classifier.
It offers a convenient way of using the objects with missing attributes to improve the accuracy of the classifier.

% CONTEXT
\chapter{Summary}

\section{Naive Bayes Classifier}

We applied the naive Bayes classifier to the insurance problem.
We saw that it performed poorly in terms of accuracy due to certain combinations of attributes and classes having zero frequencies.
This problem is common and we applied the standard approach of introducing a prior distribution for the parametrised probabilities.
After introducing this prior distribution we were able to increase our accuracy of our classifier.
We saw the accuracy of our classifier increase from 63.73\% to 69.43\%.

To use the prior distribution we had to specify the hyperparameters $\mathbf{t}$ and $s$.
The $\mathbf{t}$ hyperparameters represented our prior beliefs and we set them to be uniform.
We varied the $s$ parameter and noted that it provided a fairly consistent classifier for $s<100$ however for larger values of $s$ the accuract of our classifier decreased significantly.
This is because for large $s$ the prior dominates the likelihood and the classifier allocates classes at random.

Overall we showed that the naive Bayes classifier can be used to tackle the insurance problem despite being a relatively simple classifier.
Alternate methods of classification need to be tested to be able to put the application of the naive Bayes classifier in context.

\section{Decision Rules}

Next we looked at how we could use the ordered classes to alter the decision rule used in the naive Bayes classifier.
The problem of ordered classes falls between classification and regression.

We compared the standard 0-1 loss function to the squared difference loss function and the absolute difference loss function. 
These two difference based loss functions take into account the distance between the true class and the estimated class and assign greater loss to estimates further from the true value.

We saw that for our problem the 0-1 loss functions and the absolute difference loss function returned the same classification.
We also noted that the squared difference loss function may not return an integer classification.
We saw that if we do not round the classification we slightly reduce the mean squared error between it and the true classification.
However if we do round and measure accuracy whilst we still have a smaller mean squared error we have a lower accuracy.

Out of the three presented standard loss functions the choice of loss function depends on our objectives.
If we are looking to maximise accuracy the zero-one loss function is appropriate.
However if we want to minimize MSE the squared difference loss function could provide a better option but at the loss of accuracy.

We also looked at a custom loss matrix that can be used to fit to the insurer's needs.
We showed that a custom loss function which assigned greater loss to under estimation of risk.
This led to a cautious classifier that rarely underestimated the risk and would often overestimate the risk which led to a decrease in accuracy and increase in mean squared error.
However this may be more useful to an insurance company as it will be more costly if the underestimate the risk of a vehicle.

\section{Naive Credal Classifier}

In the next two chapters we first discussed Walley's imprecise Dirichlet model \cite{Walley96} and how we can introduce an imprecise prior to our problem.
We saw that the imprecise prior allowed us to form intervals for the probabilities we wished to estimate.
We then saw that using either the upper of lower bounds as estimates resulted in a very poor classifier.

However we were able to use these bounds to create a simple interval based classifier.
This classifier outperformed the naive Credal classifier when it was able to return a single class however it was often indeterminate and would return more that one class.

We then saw how Zaffalon's naive credal classifier performed on this problem.
We saw that this classifier is more determinate than our rudimentary interval based credal classifier however not necessarily as accurate.
We also saw that the NCC was, in some cases, too cautious when compared to the NBC.
Despite this the NCC would still be useful when classifying insurance risk.
The classifier could be used as a preliminary tool and then vehicles it is indeterminate about could be sent on to an expert.

\section{Missing Values}
Finally we saw a couple of approaches for dealing with missing values.
First we saw method of feature reduction that allowed the naive credal classifier to become more determinate but less accurate.
This is a rudimentary method that does not scale well if there are more features with missing values.

Next we used Zaffalon's approach with missing values to the naive credal classifier.
We saw that using vehicles with missing attributes improved the classifications in all the metrics we were measuring.
The ability to use missing values, a common problem when working with real world data, is a very useful feature of the naive credal classifier.

\section{Future Work}

There are a few features of our problem and data set we did not discuss in this report.

Firstly we discretized all of the continuous variables in our data set.
In the future we could look to model these variables with a distribution.
We could look at the effect using this model would have on the naive Bayes classifier and compare it to our approach of discretization.

Secondly the naivety assumption, while effective at simplifying the problem, probably is not valid for this data set.
Attributes such as width and height, and city mpg and rural mpg are in fact highly correlated.
Future research could study how to incorporate the lack of independence into both the naive Bayes classifer and the naive Credal classifier.

Finally in chapter 7 we discussed the method of multiple imputation when dealing with missing data.
It would informative to apply that method to this problem and compare its performance with the naive Credal classifier.
This would help put the naive Credal classifier in more context as the feature reduction model is not a good comparison.

\bibliography{refs}{}
\bibliographystyle{plain}

\appendix
\chapter{Automobile Dataset Attributes}\label{attributes}
\begin{tabularx}{\textwidth}{|X|l|l|l|}
	\hline
	\textbf{Name} & \textbf{Type} & \textbf{Range} & \textbf{Missing Values} \\
	\hline
	Make          & Categorical   & N/A            & YES                     \\
	Fuel Type     & Categorical   & N/A            & YES                     \\
	Fuel Type     & Categorical   & N/A            & YES                     \\
	\hline
\end{tabularx}

\end{document}