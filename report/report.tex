\documentclass[11pt]{report}

\setlength{\parindent}{0em}
\setlength{\parskip}{0.5em}

\usepackage{url}
\usepackage{amsmath}
\usepackage{cleveref}
\usepackage{ntheorem}
\usepackage[a4paper]{geometry}
\usepackage{pgfplots}
\usepackage{graphicx}
\usepackage{caption}
\usepackage{subcaption}
\usepackage{multirow}
\usepackage{tabularx}

\pgfplotsset{width=10cm,compat=1.9}

\begin{document}

\title{Credal Classification of Automobile Insurance Risk}
\author{Ben Willis}
\maketitle

\section*{Declaration}
This piece of work is a result of my own work except where it forms an assessment
based on group project work. In the case of a group project, the work
has been prepared in collaboration with other members of the group. Material
from the work of others not involved in the project has been acknowledged and
quotations and paraphrases suitably indicated.
\newpage

\begin{abstract}
	This project investigates how credal classification can be applied to the problem of determining automobile insurance risk.
	Automobiles are assigned a risk rating on a scale of -2 to 3 by an expert.
	In this project we start by building a simple naive Bayes classifier to use technical information about a vehicle to assign a risk rating.
	We find that this classifier has a moderate accuracy.
	We will then look at how the choice of loss function can be used to influence the classifier.
	We show that different choices for the loss function allow us to impreve different aspects of the naive Bayes classifier.
	Next we describe Zaffalon's naive credal classifier and apply it to our insurance problem.
	We compare our application of the naive credal classifier to the naive Bayes classifier as well as to some previous applications.
	We see that it offers a more cautious alternative which, when dealing with a small number of observations, often fails to isolate a class.
	Finally we show how the naive credal classifier can successfully make use of objects with missing values.
\end{abstract}

\tableofcontents

\chapter{Introduction}

% It's a good start. The different paragraphs don't quite flow together, e.g. there's a bit of a "jump" just after your example classifiers. Maybe use the phrase "This report is structured as follows. In Chapter 1, ..." etc. This probably should be the last part of your introduction. So, I'd suggest, for a good introduction, to stick to the following structure:

% 1. Describe your problem (could include description of data).
Classifiers have many applications in the finance industry ranging from financial trading \cite{Gerlein16} to credit card fraud detection \cite{Pozzolo15}.
In this report we will study the problem of determining the risk to an insurer of an automobile.

We have a data set containing technical information (see \cref{attributes}) and an expert assigned risk rating (on an integer scale from -2 to 3) for 205 automobiles.
The risk rating has been assigned through a process known as symboling \cite{Automobile}.
An expert considers the price of a vehicle to set an initial risk rating before moving the rating up or down depending on other attributes of the vehicle.
We want to predict the risk rating of an automobile based on the provided technical attributes.

We will tackle this problem using classification.
Classification is the problem of identifying which class an object belongs to.
Each object can be distinguished by a set of properties know as features and each object belongs to a single class.
A classifier is an algorithm which, given previous observations and their classes, can determine which class a new observation belongs to \cite{Theodoridis03}.

We want to develop a credal classifier to tackle our problem.
A credal classifier is a specific type of classifier that, instead of returning a single class, returns a set of classes.

% 2. Past work (literature that you use, similar studies).
There are many applications of classifiers including image recognition, sentiment analysis and medical diagnosis.
There are also many different approaches to the classification problem.
Firstly there are two types of classifier; supervised and unsupervised.
Supervised classifiers require a training data set describing the different classes where as unsupervised classifiers learns the classes for itself.
Some examples of classifiers include:
\begin{description}
	\item[Neural Networks] Designed to mimic the human brain it is a collection of artificial neurons which mimic the brain's axons. These neurons are then connected to each other and an input into one neuron can lead to outputs in other neurons \cite{Michie94}.
	\item[Support Vector Machines] An SVM is a supervised classifier which plots the training data in space. It then constructs planes between the classes in the data set and uses these planes to classify new observations.
	\item[Decision Trees] A decision tree is a rooted tree where each leaf represents a class and at each node a decision is made about an object. An object works its way through the tree until it reaches the end of a branch and is then assigned the corresponding class. 
\end{description}

For these problem we will build upon a simple probabilistic classifier known as the naive Bayes classifier described by Manning et al. \cite{Manning08} - chapter 13.
Manning et al present the naive Bayes classifier in the context of text classification.

We will also use the work done by Zaffalon \cite{Zaffalon01} to create our credal classifier.
Zaffalon extended Walley's imprecise Dirichlet model \cite{Walley96} to the field of classification and offered the naive credal classifier.
Zaffalon applied the classifier to multiple data sets including letter image recognition and credit ratings.
He showed that the NCC can have a greater accuracy than then NBC when it returned a single class.
He also showed that the NBC can have a significantly lower accuracy when classifying objects the NCC was indeterminate about \cite{Zaffalon01}.
We will examine whether we achieve similar results with our data set.

While there is no previous literature on classifying the insurance risk of vehicles based on this data study it has been used for other purposes.
In one study an approach to visualizing relationships between the different attributes was investigated \cite{Rosario04}.
From the visualisations created in this study we can see there are relationships between various attributes.
The classifiers we study in this report all naively assume conditional independence between the attributes given the classes so we will see whether the observed relationships affects our classifiers.

% 3. Your contribution (what you do differently from the literature, new findings).

There is plenty of literature on the naive Bayes classifier however there is relatively little on it's application to ordinal classification.
As the risk ratings we're using as our classes we can investigate ways we can alter the NBC to account for this.
We investigate alternate loss functions which take into account the distance of a misclassification.
We find we can reduce the average distance or classifications but at the expense of accuracy.

We also provide a further application of the naive Bayes classifier to this insurance problem.
As previously stated Zaffalon provided several examples of the application of the NCC to other problems.
We find that the naive Credal classifier is very accurate when returning a single class however can be quite indeterminate when faced with very few observations.

% 4. Overview of rest of report. ("This report is structured ...")
\section{Report Structure}

This report is structured as follows. In the first section we formulate a simple probabilistic classifier known as the naive Bayes classifier described.
This classifier forms the basis for the rest of the analysis in this report.
The naive Bayes classifier (or 'NBC') can be applied to many other problems including breast cancer diagnosis \cite{Dumitru09} and text analysis \cite{Nigam98}.
We start by applying the NBC to a relatively straightforward data set before applying it to the insurance problem.

In the second section we will investigate the choice of decision mechanism used in the naive Bayes classifier.
We will investigate different options that take in to account the structure of our problem.

For the remainder of the report we will expand this classifier to create Zaffalon's  naive Credal classifier (or 'NCC').


\chapter{Naive Bayes Classifier}

\section{Theory}

To demonstrate the workings of the naive Bayes classifier will introduce a new data set from a remote sensing study.
The study measured spectral information in the green, red and infrared wavelengths on three separate dates of $523$ different areas of forest in Japan.
In total we have nine continuous attributes and four possible classes: Sugi forest, Hinoki forest, Mixed deciduous forest and other non-forest land. This data set was chosen as it contained a large number of observations and demonstrates a situation where this type of classifier works well.

Formally, let us denote the class variable by $C$, taking values in the set $\{0,1,2,3,4\}$ for each of the four types of forest.
Also we measure 9 features $A_1,\dots,A_9$.
We discretize these continuous features so that they all take values in from the set $\{0,\dots,9\}$.
We denote observations of the observed values as $c$ and $a_1,\dots,a_9$ respectively.

We are interested in the probability of a forest being of type $c$ given sensor readings $\mathbf{a}$ i.e. $P(c \mid \mathbf{a})$.
Using Bayes theorem we can rewrite this as:
\begin{equation}
	P(c \mid \mathbf{a}) = \frac{P(\mathbf{a} \mid c)P(c)}{P(\mathbf{a})}
\end{equation}

Moreover we can make use of the naivety assumption.
The naivety assumptions states that each attribute is conditionally independent of one another.
In the context of this data set we are assuming the same sensor readings for the same forest are conditionally independent which indicates the possible issues with this strong assumption.
We can now write the probability of an object being in class $c$ with attributes $a_1,\dots,a_k$ as:
\begin{equation}
	P(c \mid \mathbf{a}) = \frac{P(c)\prod_{i=1}^{k}P(a_i \mid c)}{P(\mathbf{a})}
\end{equation}

To turn this into a classifier we need a way to make a decision for which class an object falls into based on the estimated probabilities.
A common method is choosing the class that maximises $P(c \mid \mathbf{a})$.
This is known as the maximum a posteriori (MAP) estimate.
We also note that $P(\mathbf{a})$ is not dependent on $C$ hence we can write our estimate as:
\begin{equation}
	\hat c = \arg\max_{c \in \mathcal{C}} P(c)\prod_{i=1}^{k}P(a_i \mid c)
\end{equation}

Now that we have our method for making our decision we need to estimate the required probabilities.

Firstly we parametrise these probabilities.
We denote the unknown chances of observing an object in class $c$ by $\theta_c$ and the chance of observing an object in class $c$ with attributes $\mathbf{a}$ by $\theta_{\mathbf{a}, c}$.
Similarly we denote the conditional chances of observing an indiviual attribute $a_i$ and a set of attributes $\mathbf{a}$ given $C=c$ by $\theta_{a_i \mid c}$ and $\theta_{\mathbf{a} \mid c}$ respectively.

Now we have parametrised the probabilities we wish to estimate we can consider the likelihood function for $\mathbf{\theta}$, the vector whose elements are the chances $\theta_{\mathbf{a}, c}$.
Using our data we denote the frequencies of objects in each class $c$ by $n(c)$ and the number of objects in class $c$ with attribute $a_i$ by $n(a_i, c)$.
For example the number of observations of class $0$ is $158$ so $n(0) = 158$
We then consider the vector $\mathbf{n}$ which contains these frequencies.

The likelihood function can be expressed as:
\begin{equation} \label{likelihood}
	l(\mathbf{\theta} \mid \mathbf{n}) \propto \prod_{c \in \mathcal{C}} \left[ \theta_c^{n(c)} \prod_{i=1}^k \prod_{a_i \in \mathcal{A}_i} \theta_{a_i \mid c}^{n(a_i, c)} \right]
\end{equation}

A simple estimate for these parameters is the maximum likelihood estimate (MLE).
To find the MLE first we take the log likelihood:
\begin{equation}
	L(\mathbf{\theta} \mid \mathbf{n}) \propto \sum_{c \in \mathcal{C}}  n(c)log(\theta_c) + \sum_{c \in \mathcal{C}} \sum_{i=1}^k \sum_{a_i \in \mathcal{A}_i} n(a_i, c) log(\theta_{a_i \mid c}) 
\end{equation}
So to maximise the likelihood function we need to maximise the each part of the log likelihood function.

To do so we use the method of Lagrange multipliers.
This is a strategy for finding local maxima and minima of a function subject to constraints.

For the $\theta{c}$ parameters we want to maximise:
\begin{equation}
	f(\mathbf{\theta}, \mathbf{n}) = \sum_{c \in \mathcal{C}}  n(c)log(\theta_c)
\end{equation}
under the constraint:
\begin{equation}
	g(\mathbf{\theta}, \mathbf{n}) = \sum_{c \in \mathcal{C}}  \theta_c - 1
\end{equation}
This gives us our Lagrangian:
\begin{equation}
	\mathcal{L}(\mathbf{\theta}, \mathbf{n}, \lambda) = \sum_{c \in \mathcal{C}}  n(c)log(\theta_c) - \lambda(\sum_{c \in \mathcal{C}}  \theta_c - 1)
\end{equation}

Differentiating with respect to $\theta_c$ we have:
\begin{equation}
	\nabla_{\theta_c} \mathcal{L}(\mathbf{\theta}, \mathbf{n}, \lambda) = \frac{n(c)}{\theta_c} - \lambda
\end{equation}

Hence the maximum likelihood estimate is $\hat{\theta_c} = \frac{n(c)}{N}$.
Intuitively this is just the relative frequency of observations that fall into that class.
Returning to our example data set we know that $N=523$ and $n(0)=158$ so $\hat\theta_0 = \frac{158}{523} \approx 0.302$

We now have our naive Bayes classifier.
We estimate $P(c)$ by $\frac{n(c)}{N}$ and $P(a_i \mid c)$ by $\frac{n(a_i, c)}{n(c)}$, the relative frequencies.
Then we choose the class $c$ which maximises $P(c)\prod_{i=1}^{k}P(a_i \mid c)$.

To measure how successful our classifier is we will initially use a technique known as $k$-fold cross validation to evaluate accuracy.
In $k$-fold cross validation we split our dataset into $k$ equally sized groups.11
Then for each group we train the classifier on all the other groups and test it on that group.
We then average all these accuracy to return an estimate for the accuracy of our classifier.

The choice of $k$ leads to different types of cross validation.
A standard choice is $k=10$. A special case of cross validation is when $k=n$ (the number of observations).
This is knowns as \textit{leave-one-out cross validation} \cite{Priddy05}.

The accuracy of the classifier is 82.37\% on this data set, using 10-fold cross validation.

\section{Application to Automobile Data set}

To make our data appropriate for this method we discretize the continuous variables into $10$ bins with an equal frequency.

Unlike in the trees data set in this data set we have objects with missing values for attributes.
We have no mechanism for considering these so we must discard these observations.
This reduces our data set from 205 observations to 193 observations.

The accuracy of the classifier is 56.48\% on this data set, using 10-fold cross validation.
This is considerably worse than the example forest data set.

\section{Conclusions}

Clearly the classifier performs better on the forest type dataset than on the auto mobile data set.
There are also general failings in our classifier we can fix to improve it form both data sets.

Firstly our classifier falls down if there are no observations with attribute $a_j$ and class $c$ in our training set. In these case the maximum likelihood estimate for $\theta_{a_j \mid c}$ is $0$.
This estimate leads to $P(c \mid \mathbf{a}) = 0$ and would rule out assigning any objects with the attribute $a_j$ to class $c$.
This is especially problematic for small sets of data.
We can tackle this by introducing prior probabilities for the theta chances.

Another reason the classifier appears to perform worse on our auto mobile data set could be the nature of its categories.
The accuracy metric does not take into account how close the classification is.
For example if the true class is 2 an assigned class of 1 should be considered better than an assigned class of -2.


\chapter{Corrected NBC with Dirichlet Prior}

\section{Theory}

We return to our likelihood function \cref{likelihood} for our theta variables.
We can introduce a prior distribution for these parameters and then consider the posterior distribution.

The Dirichlet distribution is the multinomial extension of the beta distribution for $\theta_1,\dots,\theta_k$ where $\theta_i \in (0,1)$ and $\sum_{i=1}^k \theta_i = 1$ with probability density function:
\begin{equation} \label{dirichlet_pdf}
	f(\theta_1,\dots,\theta_k \mid s, t(1),\dots,t(k)) \propto \prod_{i=1}^k \theta_i^{st(i) - 1}
\end{equation}
where $s > 0$ and each $t(i)>0$ such that $\sum_{i=1}^{k}t(i) = 1$.

We introduce a distribution that is similar to our likelihood as our prior density:
\begin{equation} \label{prior}
	f(\mathbf{\theta} \mid \mathbf{t}, s) \propto \prod_{c \in \mathcal{C}} \left[ \theta_c^{st(c) - 1} \prod_{i=1}^k \prod_{a_i \in \mathcal{A}_i} \theta_{a_i \mid c}^{st(c, a_i) - 1} \right]
\end{equation}
This is in the same form as the likelihood however each $n(\cdot)$ is replaces by $st(\cdot) - 1$.
$s > 0$ is a fixed constant and we have the following constraints on $t(\cdot):$
\begin{align}\label{prior_constraints}
	\sum_{c \in \mathcal{C}} t(c) & = 1 \\
	\sum_{a_i \in \mathcal{A}_i} t(a_i, c) & = t(c) && \forall i, c \\
	t(a_i, c) & > 0 && \forall i, a_i, c
\end{align}

This distribution is the conjugate prior for the likelihood function \cref{likelihood}.
When we multiply our likelihood by this prior density we get a posterior in the same family as this prior.
If the prior has hyper parameters $st(\cdot)$ the posterior will have hyper parameters $st(\dot) + n(\cdot)$.

We can now estimate the parameters by taking the posterior expectation e.g.
\begin{equation}
	E(\theta_c|\mathbf{n},s,\mathbf{t})=\hat{\theta_c} = \frac{n(c) + st(c)}{N + s}
\end{equation}

We estimate:
\begin{align}
	P(c) & \text{ by } \frac{n(c) + st(c)}{N + s} \\
	P(a_i \mid c) & \text{ by } \frac{n(a_i, c) + st(a_i, c)}{n(c) + st(c)}
\end{align}

\section{Application}

To apply this classifier to our data set we need to choose the hyper parameters of the prior distribution.
We note that $s$ affects the speed at which our classifier learns and $t(\cdot)$ represents our beliefs for $\theta_\cdot$.
To comply with the constraints \ref{prior_constraints} let us set:
\begin{align}
	s & = 1 \\
	t(c) & = \frac{1}{|C|} \\
	t(a_i, c) & = \frac{1}{|A_i||C|}
\end{align}

Previously we've considered the accuracy of our classifier.
An alternative metric is the mean squared error of our estimate.
\begin{equation}
	\text{MSE} = \frac{1}{n}\sum_{i=1}^n(\hat{c_i} - c_i)^2
\end{equation}

\begin{center}
	\begin{tabular}{ c|c c c c c c }
		              & Accuracy & MSE   & Failed Classifications\\
		\hline
		NBC           & 59.95\%  & 2.823 & 22.45\% \\
		Corrected NBC & 68.17\%  & 0.689 & 0\%
	\end{tabular}
\end{center}

This again shows an improvement over the previous classifier.

\section{Conclusions}
\chapter{Decision Rules}

When classifying the insurance risk of a vehicle we return a class on an integer of scale of -2 to 3.
We know that it describes order and that a vehicle with a high risk rating is of more risk to an insurer than a vehicle with a low risk rating.
However we do not know whether the intervals between the different risk ratings are equal.
Using a technique that recognises the ordinal nature of classes rather than treating them as nominal classes has been shown to improve the results of classification. \cite{Agresti10}.

In chapter two we discussed the need for a decision rule to allocate an object to a class.
We introduced the 0-1 loss function and chose the class that minimized expected loss.
We want to investigate alternate loss functions that makes use of the ordered class structure.
We also need new metrics to decide which loss functions perform better.

First we will introduce some common loss functions and the mean squared error to help determine how successful these loss functions are.
Then we will describe a custom approach using a loss matrix and see how we can affect the behaviour of our classifier through this.
Finally we will discuss some alternate approaches to ordered classification.

\section{Common Loss Functions}

In this section we outline some common loss functions based on those found in Berger \cite{Berger85}.

We will start with the 0-1 loss function as described in chapter two and then we will test alternate choices that take into account how close our estimate is to the true value.
We will use a uniform prior and take the posterior expectations to estimate the required probabilities as in the previous chapter.

\subsection{0-1 Loss Function}
Previously we considered the 0-1 loss function.
To recap this is defined by:
\begin{equation}
	L(c, \hat{c}) = 
	\begin{cases}
		0 & \text{if}\ c = \hat{c} \\
		1 & \text{otherwise}
	\end{cases}
\end{equation}

The expected loss is:
\begin{equation}
	E(L) = \sum_{c \in \mathcal{C}} L(c, \hat{c})P(c \mid \mathbf{a}) = 1 - P(\hat{c} \mid \mathbf{a})
\end{equation}

So to minimize our expected loss we choose:
\begin{equation}\label{map}
	\hat c = \arg\max_{c \in \mathcal{C}} P(c)\prod_{i=1}^{k}P(a_i \mid c)
\end{equation}
This is known as the maximum a posteriori (MAP) estimate.

\subsection{Squared Differences}
The squared differences loss function is defined as:
\begin{equation}
	L(c, \hat{c}) = (c - \hat{c})^2
\end{equation}
This assigns greater loss to risk ratings that are further away from the true value.

For this function the expected loss is:
\begin{equation}
	E(L) = \sum_{c \in \mathcal{C}} (c - \hat{c})^2P(c \mid \mathbf{a}) 
\end{equation}

Differentiating this with respect to $\hat{c}$ gives:
\begin{equation}
	\frac{\partial}{\partial \hat{c}} E(L) = \sum_{c \in \mathcal{C}} (-2c + 2\hat{c})P(c \mid \mathbf{a}) 
\end{equation}
Setting this equal to zero gives:
\begin{align}
	\sum_{c \in \mathcal{C}} cP(c \mid \mathbf{a}) & = \sum_{c \in \mathcal{C}} \hat{c}P(c \mid \mathbf{a}) \\
	E(c \mid \mathbf{a}) & = \hat{c}
\end{align}
So the estimate which minimizes the loss function is the expected class.
In the context of our problem the estimated class must be an integer, however this expected value may not be.

\subsection{Absolute Difference}
Finally we have the absolute difference loss function:
\begin{equation}
	L(c, \hat{c}) = | c - \hat{c} |
\end{equation}
Once again this assigns greater loss to risk ratings that are further away from the true value.

For this function the expected loss is:
\begin{align}
	E(L) & = \sum_{c \in \mathcal{C}} |c - \hat{c}|P(c \mid \mathbf{a}) \\
	     & = \sum_{c < \hat{c}} (\hat{c} - c)P(c \mid \mathbf{a}) - \sum_{c \geq \hat{c}} (\hat{c} - c)P(c \mid \mathbf{a})
\end{align}

Differentiating this with respect to $\hat{c}$ gives:
\begin{align}
	\frac{\partial}{\partial \hat{c}} E(L) & = \sum_{c < \hat{c}} P(c \mid \mathbf{a}) - \sum_{c \geq \hat{c}} P(c \mid \mathbf{a}) \\
	& = \sum_{c \leq \hat{c}} P(c \mid \mathbf{a}) - \sum_{c \geq \hat{c}} P(c \mid \mathbf{a})
\end{align}
Setting this equal to zero gives:
\begin{align}
	\sum_{c \leq \hat{c}} P(c \mid \mathbf{a}) & = \sum_{c \geq \hat{c}} P(c \mid \mathbf{a}) \\
	P(c \leq \hat{c} \mid \mathbf{a}) & = P(c \geq \hat{c} \mid \mathbf{a})
\end{align}
So the estimate that minimizes expected loss for this loss function is the median value.
This may be difficult to define on our data set.
For example suppose class -3 had $P(C = -2 \mid \mathbf{a}) = 0.5 = P(C=3 \mid \mathbf{a})$ then any class could reasonably be considered the median and therefore minimize our expected loss.
When more than one risk rating could be reasonably considered the median we will choose the one with the largest probability and then at random.

\section{Application}
We will now apply these loss functions to our automobile data set and measure accuracy and mean squared error.
Mean squared error is the average difference between the true class and assigned class squared.
We will also investigate how often the assigned classes agree.

As the expected posterior estimates perform better than the maximum likelihood estimates we shall use these to estimate the required probabilities.
We will also use the uniform hyperparameters and set $s=1$.

Using 10-fold cross validation our various loss functions perform as follows:

\begin{center}
	\begin{tabular}{l r r}
		\hline
		Loss Function                & Accuracy & MSE  \\
		\hline
		0-1                          & 69.99\%  & 0.59 \\
		Squared Difference           & -        & 0.55 \\
		Squared Difference (Rounded) & 67.88\%  & 0.58 \\
		Absolute Difference          & 69.99\%  & 0.59 \\
		\hline
	\end{tabular}
\end{center}

Note that the 0-1 loss function and the absolute difference loss function perform the same.
The squared difference loss function performs slightly worse in the accuracy metric after rounding however still has a lower MSE than the other loss functions.

The 0-1 loss function and absolute differences loss function assign classes in a very similar manner.
This is due to there often being a $\hat{c} \in \mathcal{C}$ with a much greater $P(c \mid \mathbf{a})$.
When this is the case $\hat{c}$ is the choice for both the 0-1 loss function and the absolute difference loss functions.

The squared difference loss function differs from these two slightly as it is affected more by outliers.

\section{Custom Loss Matrix}
In the previous section we considered three common choices for the loss function however there are other options available.

If we knew the true cost to the insurer of misclassifying a vehicle we could construct a specific loss function.
As an example consider the following loss matrix:
\begin{equation}
	L = 
		\begin{bmatrix}
			0   & 1  & 1  & 1  & 1  & 1 \\
			10  & 0  & 1  & 1  & 1  & 1 \\
			20  & 10 & 0  & 1  & 1  & 1 \\
			50  & 25 & 10 & 0  & 1  & 1 \\
			80  & 40 & 20 & 10 & 0  & 1 \\
			100 & 50 & 30 & 20 & 10 & 0 \\
		\end{bmatrix}
\end{equation}
Using this matrix we can define a loss function as $L(c, \hat{c}) = L_{c,\hat{c}}$.
In this example we assign increasing loss for underestimating the risk rating of a vehicle as this would cause the insurer to offer lower premiums despite a possibly high chance of payout.
We also assign a consistent loss of one for overestimating the risk to avoid always assigning the maximum risk rating.
This should cause our decision rule to underestimate the risk rating.
The expected loss would then be calculated as:
\begin{equation}
	E(L) = \sum_{c \in \mathcal{C}} L_{c,\hat{c}}P(c \mid \mathbf{a})
\end{equation}
The estimate that minimizes this expectation can then be chosen providing our decision rule.

If we use this particular loss function in our naive Bayes classifier and compare it to the zero one loss function we get:
\begin{center}
	\begin{tabular}{l r r}
		\hline
		Loss Function & Accuracy & MSE  \\
		\hline
		0-1           & 69.43\%  & 0.59 \\
		Custom        & 65.28\%  & 0.70 \\
		\hline
	\end{tabular}
\end{center}

Note that this custom loss function has a lower accuracy and a greater mean squared error.
However if we compare the two confusions matrices:

Zero-One:
\begin{center}
\begin{tabular}{l l r r r r r r}
    \hline
                       &    & \multicolumn{6}{c}{Predicted Class}                   \\
    \hline
                       &    & -2      & -1    & 0       & 1       & 2       & 3     \\
    \hline
\multirow{6}{*}{Actual Class} & -2 & 0.52\% & 1.04\% & 0.0\% & 0.0\% & 0.0\% & 0.0\% \\
& -1 & 1.55\% & 5.7\% & 3.63\% & 0.52\% & 0.0\% & 0.0\% \\
& 0 & 0.0\% & 1.55\% & 24.87\% & 4.66\% & 1.55\% & 0.0\% \\
& 1 & 0.0\% & 2.07\% & 2.59\% & 17.62\% & 3.11\% & 1.04\% \\
& 2 & 0.0\% & 0.0\% & 1.04\% & 2.07\% & 11.92\% & 1.04\% \\
& 3 & 0.0\% & 0.52\% & 0.0\% & 0.52\% & 2.07\% & 8.81\% \\
\hline
\end{tabular}
\end{center}

Custom Loss Matrix:
\begin{center}
\begin{tabular}{l l r r r r r r}
    \hline
                       &    & \multicolumn{6}{c}{Predicted Class}                   \\
    \hline
                       &    & -2      & -1    & 0       & 1       & 2       & 3     \\
    \hline
\multirow{6}{*}{Actual Class} & -2 & 0.0\% & 1.55\% & 0.0\% & 0.0\% & 0.0\% & 0.0\% \\
& -1 & 0.52\% & 5.7\% & 4.66\% & 0.52\% & 0.0\% & 0.0\% \\
& 0 & 0.0\% & 1.04\% & 21.76\% & 5.7\% & 4.15\% & 0.0\% \\
& 1 & 0.0\% & 1.55\% & 1.55\% & 17.1\% & 4.66\% & 1.55\% \\
& 2 & 0.0\% & 0.0\% & 1.04\% & 2.07\% & 11.92\% & 1.04\% \\
& 3 & 0.0\% & 0.52\% & 0.0\% & 0.52\% & 2.07\% & 8.81\% \\
\hline
\end{tabular}
\end{center}

We see that our custom loss function is much less likely to underestimate the risk rating of a vehicle.
This may be more beneficial to an insurer than a more accurate classifier that sometimes underestimates the risk.

\section{Conclusions}

% Implicitly restate your thesis/position.
We have found that using an alternative loss function to create the decision rule can lead to increases in some metrics.
We saw that using the squared differences loss function reduces the mean squared error but at the expense of accuracy.
We also saw how a custom choice of loss matrix can affect our classifications.

% Emphasize the importance of your subject by placing it in a larger context.
Ordered classification falls between ordinary classification and regression and has not been fully explored despite it being a common occurrence in the real world.
We have demonstrated that varying the loss function allows us to achieve different goals.

% Offer suggestions for the future based on what you have argued.
Further work on identifying how to deal with ordered classes is required given their prominence.
An alternate approach to ordinal classification was described by Frank and Hall \cite{Frank01}.
They proposed reducing the classification to a series of binary classifications.
For our insurance problem this would involved deciding whether the vehicle's true risk rating is greater than each possible risk rating.
Then after making each of these judgements we are able to assign a class.
They showed this method can lead to a more accurate classifier than one that ignores ordering when handling ordinal classes.
\newcommand{\sn}[2]{\ensuremath{{#1}\times 10^{#2}}}

\chapter{Imprecise Prior}

When estimating the probabilities for our classifier we take into account our prior beliefs for them and the likelihood given a set of observations.
When choosing our prior distribution we had to pick hyperparameters without any knowledge.
In this chapter we will use Walley's imprecise Dirichlet model \cite{Walley96} to model this lack of knowledge and then create a simple interval based credal classifier.

\section{Imprecise Dirichlet Model}

When we initially chose our prior distribution we chose the hyperparameters in \cref{initial prior} such that they follow the principle of indifference.
However as previously mentioned this prior does not truly represent a lack of prior knowledge.
We need a way to represent this lack of knowledge and acknowledge that our estimates for the probabilities depend on our choice of hyperparameters.

The imprecise Dirichlet model is a model for this lack of knowledge introduced by Walley \cite{Walley96}.
Instead of using a single prior distribution to represent our beliefs about the unknown parameters we use a set of prior distributions.

In the imprecise Dirichlet model the prior distributions are Dirichlet distributions with parameters $(s, \mathbf{t})$ such that $\sum_{c \in \mathcal{C}} t(c) = 1$.
The distributions for the likelihood are multinomial so:
\begin{equation}
	f(\mathbf{n} \mid \bm{\theta}) \propto \prod_{c \in \mathcal{C}} \theta_c^{n(c)}
	\qquad
	f(\bm{\theta} \mid s, \mathbf{t}) \propto \prod_{c \in \mathcal{C}} \theta_c^{st(c) - 1}
\end{equation}
Then the posterior distribution is of the form:
\begin{equation} \label{dirichlet_pdf2}
	f(\mathbf{\theta} \mid \mathbf{n}, s, \mathbf{t}) \propto \prod_{c \in \mathcal{C}} \theta_c^{n(c) + st(c) - 1}
\end{equation}
which is also a Dirichlet distribution with parameters $(N+s, \frac{\mathbf{n}+s\mathbf{t}}{N+s})$.
We can then take the posterior expectation for the $\bm{\theta}$ chances giving:
\begin{equation}
	P_t(c) = E(\theta_c \mid \mathbf{n}, s, \mathbf{t}) = \frac{n(c)+st(c)}{N+s}
\end{equation}
Note this is a function of $t(c)$ which allow us to obtain upper and lower estimates for for probabilities by varying $t(c)$:
\begin{align}
	\overline{P}(c) & = \frac{n(c)+s}{N+s} & (t(c) \rightarrow 1) \\
	\underline{P}(c) & = \frac{n(c)}{N+s}  & (t(c) \rightarrow 0)
\end{align}

Zaffalon applied this method of imprecise priors to the problem of classification.
We will use a set of the prior distributions of the form in \cref{prior} for a fixed value of $s$.
The parameter $s$ represents the strength of our prior beliefs and determines how quickly our classifier learns.

\section{Imprecise Probabilities}

Imprecise probability refers to the partial specification of a probability for example through upper and lower bounds for a probability.
Imprecise probabilities have other applications in artificial intelligence and can better represent an experts knowledge \cite{Coolen11}.
Using the model of imprecise priors we can find the upper and lower bounds for each posterior expectation based on different prior distributions.

Recall that we estimate the probabilities by taking the expectation of the posterior distribution and that these estimates depend on $\mathbf{t}$ i.e.:
\begin{align}
	P_t(c) & = E(\theta_c \mid \mathbf{n},s,\mathbf{t}) = \frac{n(c) + st(c)}{N + s} \\
	P_t(a_i \mid c) & = E(\theta_{a_i \mid c} \mid \mathbf{n},s,\mathbf{t}) = \frac{n(a_i, c) + st(a_i, c)}{n(c) + st(c)}
\end{align}
We can then use these to find upper and lower estimates for the probabilities over all values of $\mathbf{t}$ in our prior model.
For our distributions the upper and lower bounds are given by:
\begin{align}
	\overline{P}(c) & = \frac{n(c) + s}{N+s} \\
	\underline{P}(c) & = \frac{n(c)}{N+s}
\end{align}
These occur when we use the prior distributions with $t(c) \rightarrow 0$ and $t(c) \rightarrow 1$ respectively.

Similarly we have:
\begin{align}
	\overline{P}(a_i \mid c) & = \frac{n(a_i, c) + s}{n(c)+s} \\
	\underline{P}(a_i \mid c) & = \frac{n(a_i, c)}{n(c)+s}
\end{align}
These occur when we use the prior distributions with $t(c) \rightarrow 1$, $t(a_i, c)\rightarrow1$ and $t(c) \rightarrow 1$, $t(a_i, c)\rightarrow0$ respectively.

Let's start by comparing how our classifier behaves if we assume the true probability is at each end of the interval.
We will use the 0-1 loss function for the decision method and estimate each probability $P(\cdot)$ by either the upper or lower probability of our interval.
We will measure accuracy and indeterminate classifications as before.

\begin{center}
	\begin{tabular}{l|c c}
	                & Accuracy & Indeterminate Assignments \\
	\hline
	Lower Estimates & 62.17\%  & 20.02\%            \\
	Upper Estimates & 40.09\%  & 0\%                \\
	\end{tabular}
\end{center}

Neither of these offer a sufficient classification.
There are often no observations of an attribute with a particular class which is why using the lower estimates leads to indeterminate classifications despite a reasonable accuracy for this data set.
On the other hand when using the upper estimates our classifier has very low accuracy.

\section{Simple Credal Classifier}
Alternatively we can turn our classifier into a credal classifier.
A credal classifier assigns a set of classes as opposed to a single class to an object.

Recall the MAP estimate for the risk rating of a vehicle given by \cref{map}.
We say that a class $c'$ is dominated by $c''$ if:
\begin{equation}\label{Credal Dominance}
	P_t(c' \mid \mathbf{a}) < P_t(c'' \mid \mathbf{a})
\end{equation}
for all values of $\mathbf{t}$ in our prior model.
This is because the MAP estimate for the class will always choose $c''$ over $c'$ regardless of which prior is used.
Note that this is equivalent to $P_t(c', \mathbf{a}) < P_t(c'', \mathbf{a})$ as $P(\mathbf{a})$ does not depend on $c$.

We can use the bounds for the probabilities we found earlier to create an interval $P(c, \mathbf{a})$ must lie in. 

If we look at the intervals created by the upper and lower estimates we achieve:
\begin{equation}
	P_t(c, \mathbf{a}) \in \left[ \underline{P}(c)\prod_{i=1}^k \underline{P}(a_i \mid c), \overline{P}(c)\prod_{i=1}^k \overline{P}(a_i \mid c) \right]
\end{equation}
for each $c \in \mathcal{C}$.
This is true because:
\begin{equation}
\underline{P}(c)\prod_{i=1}^k \underline{P}(a_i \mid c) \leq \underline{P}(c, \mathbf{a}) \leq P_t(c, \mathbf{a}) \leq \overline{P}(c, \mathbf{a}) \leq \overline{P}(c)\prod_{i=1}^k \overline{P}(a_i \mid c)
\end{equation}
for all choices of $\mathbf{t}$.
We can use the above intervals to create a simple credal classifier.

For an example consider the following intervals:
\begin{center}
	\begin{tabular}{l|c c}
	Risk Rating & Lower Bound & Upper Bound \\
	\hline
	-2          & $0$              & $\sn{3.12}{-9}$  \\
	-1          & $0$              & $\sn{8.91}{-16}$ \\
	0           & $\sn{3.81}{-15}$ & $\sn{2.30}{-13}$ \\
	1           & $\sn{2.19}{-9}$  & $\sn{2.34}{-8}$  \\
	2           & $\sn{1.88}{-13}$ & $\sn{6.22}{-11}$ \\
	3           & $0$              & $\sn{5.82}{-17}$ \\
	\end{tabular}
\end{center}
We can see the classes -1, 0, 2 and 3 are dominated by 1.
However the risk rating of -2 is not dominated by 1 as $\sn{3.12}{-9} > \sn{2.19}{-9}$.
Hence our credal classifier returns the set of risk ratings $\{-2, 1\}$.
Note that in this particular example the true risk rating was 1 so our credal classifer was correct to include it in the set of possible classes.

\section{Diagnostics}

We now need a way to test this classifier.
We cannot use our previous measure of accuracy as this classifier may not return a single class.
Instead there are a few metrics we can use for our diagnostics \cite{Antonucci11}:
\begin{description}
	\item[Single Accuracy (A\%)] Accuracy of the credal classifier when a single class is returned
	\item[Set Accuracy (B\%)] Percentage of objects for which the true class is in the returned set when the set has size larger than one
	\item[Indeterminate Output Size (C)] Average set size for returned sets containing more than one class
	\item[Determinacy (D\%)] Percentage of objects for which the returned set contains one class
\end{description}

\section{Application}

We will vary the choice of the hyperparameter $s$ when measuring these three statistics to see its effect.

%Seed = 0.1

\begin{center}
\begin{tabular}{l|c c c c}
        & A\%     & B\%     & C    & D\%     \\
\hline
s = 0.5 & 77.02\% & 87.05\% & 3.71 & 38.34\% \\
s = 1   & 76.92\% & 90.67\% & 3.75 & 20.20\% \\
s = 2   & 75.00\% & 94.30\% & 4.04 & 3.11\% \\
s = 5   & -       & 97.92\% & 5.22 & 0\%   \\
\end{tabular}
\end{center}

We see that the single accuracy of our classifier slightly decreases for the different $s$ parameters.
We also note that this single accuracy is greater than the accuracy of our corrected naive Bayes classifier on the same data set.
However we notice that varying the $s$ parameter has an effect on the other two metrics.
Increasing the value of $s$ decreases determinacy and increases the set accuracy.

This effect can be easily explained.
Increasing the $s$ value increases the upper bound and decreases the lower bound on each of the probabilities being estimated.
Hence increasing the value of $s$ increases the size of the interval and increasing the size of the interval leads to less intervals being dominated and fewer classes being excluded.

\section{Conclusion}

In this chapter we have seen how Walley's imprecise Dirichlet model can be used to create intervals for the probabilities we wish to estimate when using the naive Bayes classifier.
We have seen that using either the upper and lower bounds lead to a very poor classifier.
We then used these imprecise probabilities to create a simple credal classifier.
\chapter{The Naive Credal Classifier}

In the simple credal classifier we estimated the lower and upper bounds for each probability separately using our imprecise prior model.
We then used these separate estimates to make inferences about the true probability of interest: $P(c \mid \mathbf{a})$.

However an alternate method for credal classification was proposed by Zaffalon \cite{Zaffalon01} which we will outline in this section.
This classifier is known as the naive Credal classifier (NCC) and can be more determinate than the simple credal classifier we studied earlier.

\section{Previous Work}

Zaffalon has provided multiple examples of this classifier in action.

In one such study he applies the naive Credal classifier to dementia diagnosis \cite{Zaffalon03}.
He starts with a data set containing test results for 3385 different patients split in to five different categories (four describing types of dementia sufferers, one describing healthy patients).
The data set also contains missing values for some of the patient's test results.
In the first part of the study he simply distinguishes between dementia sufferers and non-dementia sufferers.
In this part the NCC is able to isolate a single class about 90\% of the time and in these cases is accurate 95\% of the time.
When it fails to isolate a single class the NBC is only able to classify the same object correctly 70\% of the time.
In the second part he limits the study to the type of dementia thus reducing the data set to 1103 observations.
Here we see similarly positive results; when the NCC isolates a single class it is accurate 94\% of the time and when it outputs more than one class the set size is about 2 on average and the true class is in the set 98\% of the time.
This study gives an example when a large data set can lead to the NCC being highly determinate and accurate.

In another study he applies the NCC to environmental mining data \cite{Zaffalon02}.
Here the data falls in to one of four categories however the data set only consists of 155 complete instances which is much closer in size to our insurance problem.
In this study the NCC only produced a single class 60\% of the time and of these classes had a single accuracy of 52\%.
When returning more than one class the true class was contained in the output set 82\% of the time.
In comparison the NBC had an accuracy of 48\% on the whole data set and 43\% on the subset of objects the NCC was indeterminate about.
This provides a good example of a situation where the NCC will withhold judgement due to a lack of information.

\section{Theory}

To define the naive Credal classifier we first rewrite our original definition of credal dominance \cref{Credal Dominance} as:
\begin{equation}
	\frac{P_t(c' \mid \mathbf{a})}{P_t(c'' \mid \mathbf{a})} = \frac{P_t(c')\prod_{i=1}^{k}P_t(a_i \mid c')}{P_t(c'')\prod_{i=1}^{k}P_t(a_i \mid c'')} > 1
\end{equation}
Note that the equality holds because the constant $P(\mathbf{a})$ is cancelled.

If we use the posterior expectation for our parametrisation as an estimate for the probabilities then we arrive at:
\begin{equation}
	\frac{n(c')+st(c')}{n(c'')+st(c'')} \prod_{i=1}^k \frac{n(a_i, c') + st(a_i , c')}{n(c'') + st(c'')} \frac{n(c'') + st(c'')}{n(a_i, c') + st(a_i , c')} > 1
\end{equation}

To determine whether this inequality holds we can solve the optimization problem:
\begin{align}
	\min & \left[ \frac{n(c'')+st(c'')}{n(c')+st(c')} \right]^{k-1} \prod_{i=1}^k \frac{n(a_i, c') + st(a_i , c')}{n(a_i, c'') + st(a_i , c'')} \\
	\text{s.t.} & \sum_{c \in \mathcal{C}} t(c) = 1 \\
	& 0 < t(a_i, c) < t(c)
\end{align}
Then compare the answer to 1.
This is the same optimization problem as described by Zaffalon \cite{Zaffalon01}.

It is possible to manipulate this problem into an format that is is easier to solve.
Firstly note that the minimum is achieved when each $t(a_i, c') \rightarrow 0$ and $t(a_i, c'') \rightarrow t(c'')$ so we can use these values in the objective function.
Furthermore, at the minimum, we have $t(c') = 1 - t(c'')$.
To simplify the problem set $st(c'') = x$ then our optimization problem becomes a problem in a single variable:
\begin{align} \label{Credal Dominance Test}
	\min \quad & f(x) = \left[ \frac{n(c'') + x}{n(c') + s - x} \right]^{k-1} \prod_{i=1}^k \frac{n(a_i, c')}{n(a_i, c'') + x} \\
	\text{s.t.} \quad & 0 < x < s
\end{align}

Before we solve this optimization problem we can rule out an edge case.
If $n(a_i, c')=0$ for any $a_i$ then the $c'$ does not dominate $c''$.
If $n(a_i, c'')=0$ for any $a_i$ then we set $f(0)=10$ to indicate domination is achieved at this point.

Next step is to figure out what the objective function looks like.
Note that it is always positive so if we take the log of $f$ and differentiate we get:
\begin{equation}
	\frac{d\ln(f)}{dx} = \frac{k-1}{n(c'')+x} + \frac{k-1}{n(c')+1-x} - \sum_{i-1}^k \frac{1}{n(a_i, c'') + x}
\end{equation}
Differentiating again gives:
\begin{equation}
	\frac{d^2\ln(f)}{dx^2} = -\frac{k-1}{(n(c'') + x)^2} + \frac{k-1}{(n(c')+1-x)^2} + \sum_{i=1}^k \frac{1}{(n(a_i, c'') + x)^2}
\end{equation}
This is always positive so the objective function is log concave.
As the logarithm is monotone it follows that $f(x)$ is also concave and hence has a single minimum.

If we remove the edge cases described above we are left with the simple problem of finding the maximum of a concave function in a given interval.
To do so we use the fminbound function in the SciPy library which in turn uses Brent's method \cite{fminbound}.

\section{Application}

We will measure the same metrics as previously for this new classifier.
The results from 10-fold cross validation with varying $s$ values are as follow:

\begin{tikzpicture}
\begin{axis}[
    xlabel={s},
    ylabel={Percentage \%},
    xmin=0, xmax=5.5,
    ymin=0, ymax=100,
	legend pos=outer north east
]

\addplot[
    color=blue,
    mark=square,
    ]
    coordinates {
    (0.1,79.51)(0.5,77.03)(1,75.56)(2,76.92)
    };
    \label{sng_acc}

\addplot[
    color=red,
    mark=square,
    ]
    coordinates {
    (0.1,86.52)(0.5,88.60)(1,88.60)(2,90.16)(3, 91.19)(4, 92.23)(5,93.26)
    };
    \label{set_acc}
 
\addplot[
    color=green,
    mark=square,
    ]
    coordinates {
    (0.1,63.21)(0.5,38.32)(1,23.31)(2,6.74)(3, 0)(4, 0)(5,0)
    };
    \label{det}

\addlegendimage{/pgfplots/refstyle=sng_acc}\addlegendentry{Single Accuracy}
\addlegendimage{/pgfplots/refstyle=set_acc}\addlegendentry{Set Accuracy}
\addlegendimage{/pgfplots/refstyle=det}\addlegendentry{Determinacy}
\end{axis}

\end{tikzpicture}

Additionally we have the following measurements for indeterminate output size:
\begin{center}
\begin{tabular}{c|c c c c}
s & 0.5 & 1 & 2 & 5 \\
\hline
Indeterminate Output Size & 3.65 & 3.66 & 3.74 & 4.25
\end{tabular}
\end{center}

First we note the increase in indeterminate output size and decrease in determinacy.
These are due to the domination criteria becoming harder to satisfy for larger $s$ values.
This leads to less classes being credal dominated and larger output size.

This can also explain the slight trends in single and set accuracy.
Set accuracy increases because the indeterminate output size is always increasing so for each increase in $s$ value we are more likely to see the true class added to the output set if it was not already there.
On the other hand the single accuracy does not change much.
As we increase the $s$ value we decrease the number of single outputs and it would appear the outputs that become indeterminate are equally likely to be correct classifications as incorrect classifications.

We can also directly compare the classifications of the naive Credal classifier to those of our interval based classifier. For $s=1$ we have:
\begin{center}
\begin{tabular}{l|c c c c}
         & A\%     & B\%     & C    & D\%     \\
\hline
Interval & 75.61\% & 89.12\% & 3.71 & 21.32\% \\
NCC      & 75.56\% & 88.60\% & 3.66 & 23.31\% \\
\end{tabular}
\end{center}
Here we see similar set and single accuracies.
However we see that the naive Credal classifier is more determinate than the interval based classifier and, when indeterminate, has a smaller average output size.

Our data set only contains observations of three vehicles with risk rating -2.
This means that our credal classifiers struggle to eliminate this classification as an option.
When we consider situations where the NCC is indeterminate and returns two possible classes -2 is always one of these classes.
Additionally the other class is the correct classification 88.60\% of the time.
These situations demonstrate how the NCC will be indeterminate when there are a small number of observations of a particular class.

\section{Comparison to NBC}

In addition to our interval based credal classifier we can also compare the NCC to the NBC.
We will do this using the same statistics as Zaffalon \cite{Zaffalon01}.
Zaffalon compared the single accuracy of the NCC to the accuracy of the NBC.
He also considered the accuracy of the NBC limited to objects the NCC was indeterminate about.
We will set $s=1$ and consider the same three metrics: NCC single accuracy (A\textsubscript{NCC}\%), NBC accuracy A\textsubscript{NBC} and NBC accuracy on  the NCC's indeterminate objects (A\textsubscript{S}\%).
\begin{center}
\begin{tabular}{c c c}
\hline
A\textsubscript{NCC}\% & A\textsubscript{NBC} & A\textsubscript{s}\% \\
\hline
81.40\%                & 69.43\%              & 66.00\% \\
\hline
\end{tabular}
\end{center}

This demonstrates how the indecisive nature of the naive Credal classifier can hold it back.
We see that when it returns a single class it is far more accurate than the naive Bayes classifier.
However we also see that for objects the NCC is indeterminate about the NBC is only slightly less accurate.
These measurements are in contrast to those achieved by Zaffalon.
When he carried out his analysis he saw that the NBC drop off significantly when only considering the classes the NCC was indeterminate about.

The reason for this difference again comes back to the lack of observations of vehicles with a -2 risk rating.
A lot of the time when the NCC is indeterminate it is only with regards to the true class and the -2 risk rating.
The NBC is more decisive and will opt for the correct class in many of these occasions.

\section{Conclusions}

We have seen that Zaffalon's naive Credal classifier is slightly more determinate than our simple interval based classifier.
We have also seen how our choice of the $s$ hyper parameter affects how cautious our classifier is.

However we have also seen that the naive Credal classifier can be very cautious when there are a small number of observations of a particular class.
We saw that when this is the case it would often return sets of size two containing the true class and the class with a low number of observations.
We saw that this meant the NBC still had a good accuracy when applied to the sets the NCC was indeterminate about because it is a less cautious classifier.
\chapter{Missing Attributes}

Previously we had no method for classifying or using the vehicles in our data set with missing attributes.
However there are methods we can apply to overcome this issue which we will discuss in this chapter.
Note that in our data set the class is never missing from a vehicle so we only have to deal with missing technical attributes.

\section{NCC Approach}

One such method builds upon our test for credal dominance in \cref{Credal Dominance}.
We can factor in our our uncertainty of the frequencies by giving lower and upper bounds for them:
\begin{align}
	\min \quad & f(x) = \left[ \frac{n(c'') + x}{\underline{n}(c') + 1 - x} \right]^{k-1} \prod_{i=1}^k \frac{n(a_i, c')}{\overline{n}(a_i, c'') + x} \\
	\text{s.t.} \quad & 0 < x < s
\end{align}
These lower and upper bounds for $n(a_i \mid c)$ are found by considering all possibilities for the missing attributes.
We can find the solution to this optimization problem as before.

We will apply this method to our data set and compare it to previous results.
Unlike before we will not vary the $s$ value and instead use $s=1$ throughout.
As before we will measure four metrics: Single Accuracy (A\%), Set Accuracy (B\%), Indeterminate Output Size (C) and Determinacy (D\%).

\begin{center}
\begin{tabular}{l|c c c c}
                       & A\%     & B\%     & C    & D\%     \\
\hline
Without missing values & 79.06\% & 90.15\% & 3.79 & 22.27\% \\
With missing values    & 78.94\% & 89.76\% & 3.56 & 19.02\% \\
\end{tabular}
\end{center}

We see similar results for single and set accuracy which is a positive sign considering we are not classifying objects are missing attributes.
However there is a small drop in determinacy when using vehicles with missing attributes.
This could be due to the uncertainty in our frequency counts leading to less classes being dominated.
This would also explain the decrease in indeterminate output size.
\chapter{Conclusion}

In this chapter we will summarise our findings and put them in to context.

\section{Naive Bayes Classifier}

We applied the naive Bayes classifier to the insurance problem.
We saw that it performed poorly in terms of accuracy however we were able to improve the accuracy by introducing a prior distribution for the parametrised probabilities.

When we set $s=0.01$ we achieved an accuracy of 74.61\% which is alright (comparable to other studies?).
This indicates that while the NBC could be used for the purposes of classifying insurance risk it may not be the most effective method.

\section{Loss Function}

We varied the loss function used to make the decision in the naive Bayes classifier.
We compared the standard 0-1 loss function to the squared difference loss function and the absolute difference loss function.

We saw that for our problem the 0-1 loss functions and the absolute difference loss function returned the same classification.
We also noted that the squared difference loss function may not return an integer classification.
We saw that if we don't round the classification we slightly reduce the mean squared error between it and the true classification.
However if we do round and measure accuracy whilst we still have a smaller mean squared error we have a lower accuracy.

Overall there is little reason to not use the 0-1 loss function for this particular problem problem.

\section{Naive Credal Classifier}

First of all we discussed Walley's imprecise Dirichlet model and how the we can introduce an imprecise prior to our problem.
We created a rudimentary credal classifier by considering upper and lower probabilities derived from this imprecise model and looking at interval dominance.
We then used the imprecise prior to follow Zaffalon's formulation of the naive Credal classifier.
We applied this classifier to our problem with mixed results.
We saw that this classifier is more determinate than our rudimentary classifier however not necessarily as accurate.

Despite this the NCC would still be useful when classifying insurance risk.
For $s=0.1$ when it returns a single class it has an accuracy of 80\%.
The classifier could be used as a preliminary tool and then vehicles it is indeterminate about could be sent on to an expert.

\section{Future Work}

There are a few features of our problem and data set we did not discuss in this report.

Firstly we discretized all continuous variables.
In the future we could look to model these variables with some kind of distribution.

Secondly the naivety assumption, while effective at simplifying the problem, probably is not valid for this data set.
Attributes such as width and height, and city mpg and rural mpg are in fact highly correlated and not conditionally independent.
Future research could study how to incorporate the lack of independence into the naive Credal classifier.

\include{bibliography}

\appendix
\chapter{Automobile Dataset Attributes}\label{attributes}
\begin{tabularx}{\textwidth}{|X|l|l|l|}
	\hline
	\textbf{Name} & \textbf{Type} & \textbf{Range} & \textbf{Missing Values} \\
	\hline
	Make          & Categorical   & N/A            & YES                     \\
	Fuel Type     & Categorical   & N/A            & YES                     \\
	Fuel Type     & Categorical   & N/A            & YES                     \\
	\hline
\end{tabularx}

\end{document}