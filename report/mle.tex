\chapter{MLE estimate for NBC}

First we look at the maximum likelihood estimate for the naive Bayes classifier.

\section{Notation}

Formally, let us denote the class variable by $C$, taking values in the set $\mathcal{C}$. Also we measure $k$ features $A_1,\dots,A_k$ from the sets $\mathcal{A}_1,\dots,\mathcal{A}_k$. We denote observations of these variables as $c$ and $a_1,\dots,a_k$ respectively.

We will also denote the unknown chances of observing an object with $C=c$ by $\theta_c$ and the change of observing an object with $C=c$ and $\mathbf{A} = \mathbf{a}$ by $\theta_{\mathbf{a}, c}$. Similarly we denote the conditional chances of $A_i=a_i$ and $(A_1,\dots,A_k)=(a_1,\dots,a_k)$ given $C=c$ by $\theta_{a_i \mid c}$ and $\theta_{\mathbf{a} \mid c}$ respectively.

Finally after making observations of the attributes and class of $N$ objects. We denote the frequency of those in class $c$ by $n(c)$ and those in class c with attribute $a_i$ by $n(a_i, c)$. We have the following structural constraints:
\begin{gather}
	0 \leq n(a_i \mid c) \leq n(c) \\
	\sum_{a_i \in \mathcal{A}_i} n(a_i | c) = n(c) \\
	\sum_{c \in \mathcal{C}} n(c) = N
\end{gather}
\section{Assumptions}

Both the NCC and the NBC share the naivety assumption \cite{Zaffalon01}. This is the assumption that the features of an object are independent \cite{Rish01}. Hence:
\begin{equation} \label{naive_assumption}
	\theta_{\mathbf{a} \mid c} = \prod_{i=1}^{k} \theta_{a_i \mid c}
\end{equation}
This assumption greatly simplifies the problem.

They also make use of Bayes' theorem which allows us to rewrite the probability of an object belonging to a class like so:
\begin{equation} \label{bayes_theorem}
    P(A \mid B) = \frac{P(B \mid A) P(A)}{P(B)}
\end{equation}

\section{Likelihood Function}

Using \cref{bayes_theorem,naive_assumption} we can derive the likelihood function for the $\mathbf{\theta}$, the vector whose elements are the chances $\theta_{\mathbf{a}, c}$ given data $\mathbf{n}$, the vector of all known frequencies.

The likelihood function can be expressed as:
\begin{equation} \label{likelihood}
	l(\mathbf{\theta} \mid \mathbf{n}) \propto \prod_{c \in \mathcal{C}} \left[ \theta_c^{n(c)} \prod_{i=1}^k \prod_{a_i \in \mathcal{A}_i} \theta_{a_i \mid c}^{n(c, a_i)} \right]
\end{equation}

\section{Maximum Likelihood Estimate}

We can derive the maximum likelihood estimate from this function.

First we take the log likelihood:
\begin{equation}
	L(\mathbf{\theta} \mid \mathbf{n}) \propto \sum_{c \in \mathcal{C}}  n(c)log(\theta_c) + \sum_{c \in \mathcal{C}} \sum_{i=1}^k \sum_{a_i \in \mathcal{A}_i} n(c, a_i) log(\theta_{a_i \mid c}) 
\end{equation}
So to maximise the likelihood function we need to maximise the two parts of the log likelihood function.

To do so we use the method of Lagrange multipliers. This is a strategy for finding local maxima and minima of a function subject to constraints.

For the first equation we have
\begin{align}
	f(\mathbf{\theta}, \mathbf{n})& = \sum_{c \in \mathcal{C}}  n(c)log(\theta_c) \\
	g(\mathbf{\theta}, \mathbf{n})& = \sum_{c \in \mathcal{C}}  \theta_c - 1
\end{align}
This gives us our Lagrangian:
\begin{equation}
	\mathcal{L}(\mathbf{\theta}, \mathbf{n}, \lambda) = \sum_{c \in \mathcal{C}}  n(c)log(\theta_c) - \lambda(\sum_{c \in \mathcal{C}}  \theta_c - 1)
\end{equation}

Differentiating with respect to $\theta_c$ we have:
\begin{equation}
	\nabla_{\theta_c} \mathcal{L}(\mathbf{\theta}, \mathbf{n}, \lambda) = \frac{n(c)}{\theta_c} - \lambda
\end{equation}

Hence the maximum is achieved giving an mle of $\hat{\theta_c} = \frac{n(c)}{N}$. Intuitively this is just the relative frequency of observations that fall into that class.

\section{Classification}

We can use the maximum likelihood estimates for these chances to create a naive Bayes Classifier. We can estimate $P(c|\mathbf{a})$ with our maximum likelihood estimates for our theta chances.

Applying this to our data set and using a technique known as $k$-fold cross validation to evaluate accuracy. In $k$-fold cross validation we split our dataset into $k$ equally sized groups. Then for each group we train the classifier on all the other groups and test it on that group. We then average all these accuracy to return an (unbiased?) estimate for the accuracy of our classifier.

The choice of $k$ leads to different types of cross validation. A standard choice is $k=10$. A special case of cross validation is when $k=n$ (the number of observations). This is knowns as \textit{Leave-one-out cross validation} \cite{Priddy05}.

In this case we set $k=10$ and the estimated accuracy was \%66.54.

This is pretty bad, can we improve?