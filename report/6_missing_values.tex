\chapter{Missing Attributes}

Previously we had no method for classifying or using the vehicles in our data set with missing attributes.
However there are methods we can apply to overcome this issue which we will discuss in this chapter.
Note that in our data set the class is never missing from a vehicle so we only have to deal with missing technical attributes.

\section{NCC Approach}

One such method builds upon our test for credal dominance in \cref{Credal Dominance}.
We can factor in our our uncertainty of the frequencies by giving lower and upper bounds for them:
\begin{align}
	\min \quad & f(x) = \left[ \frac{n(c'') + x}{\underline{n}(c') + 1 - x} \right]^{k-1} \prod_{i=1}^k \frac{n(a_i, c')}{\overline{n}(a_i, c'') + x} \\
	\text{s.t.} \quad & 0 < x < s
\end{align}
These lower and upper bounds for $n(a_i \mid c)$ are found by considering all possibilities for the missing attributes.
We can find the solution to this optimization problem as before.

We will apply this method to our data set and compare it to previous results.
Unlike before we will not vary the $s$ value and instead use $s=1$ throughout.
As before we will measure four metrics: Single Accuracy (A\%), Set Accuracy (B\%), Indeterminate Output Size (C) and Determinacy (D\%).

\begin{center}
\begin{tabular}{l|c c c c}
                       & A\%     & B\%     & C    & D\%     \\
\hline
Without missing values & 79.06\% & 90.15\% & 3.79 & 22.27\% \\
With missing values    & 78.94\% & 89.76\% & 3.56 & 19.02\% \\
\end{tabular}
\end{center}

We see similar results for single and set accuracy which is a positive sign considering we are not classifying objects are missing attributes.
However there is a small drop in determinacy when using vehicles with missing attributes.
This could be due to the uncertainty in our frequency counts leading to less classes being dominated.
This would also explain the decrease in indeterminate output size.