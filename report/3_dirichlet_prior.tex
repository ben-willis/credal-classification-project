\chapter{Corrected NBC with Dirichlet Prior}

\section{Theory}

We return to our likelihood function \cref{likelihood} for our theta variables.
We can introduce a prior distribution for these parameters and then consider the posterior distribution.

The Dirichlet distribution is the multinomial extension of the beta distribution for $x_1,\dots,x_k$ where $x_i \in (0,1)$ and $\sum_{i=1}^k x_i = 1$ with probability density function:
\begin{equation} \label{dirichlet_pdf}
	f(x_1,\dots,x_k \mid s, t(1),\dots,t(k)) \propto \prod_{i=1}^k x_i^{st(i) - 1}
\end{equation}
where $s > 0$ and each $t(i)>0$ such that $\sum_{i=1}^{k}t(i) = 1$.

We introduce a similar distribution as our prior density:
\begin{equation} \label{prior}
	f(\mathbf{\theta} \mid \mathbf{t}, s) \propto \prod_{x \in \mathcal{C}} \left[ \theta_c^{st(c) - 1} \prod_{i=1}^k \prod_{a_i \in \mathcal{A}_i} \theta_{a_i \mid c}^{st(c, a_i) - 1} \right]
\end{equation}
where $t(\cdot)$ corresponds to $n(\cdot)$.
This prior Dirichlet distribution \cite{Zaffalon01} has the following constraints:
\begin{equation}
	\sum_{c \in \mathcal{C}} t(c) = 1
\end{equation}
\begin{equation}
	\sum_{a_i \in \mathcal{A}_i} t(a_i, c) = t(c)
\end{equation}
\begin{equation}
	t(a_i, c) > 0
\end{equation}
For all $(i, a_i, c)$.

When we multiply our likelihood by this prior density get a posterior in the same form.
This prior distribute is same conjugate family as our likelihood function.

How do we choose prior parameters? $s$ affects the speed at which our classifier learns and $t(c)$ kind of represents our beliefs for $\theta_c$.