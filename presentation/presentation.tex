\documentclass{beamer}
 
\usepackage[utf8]{inputenc}
\usepackage{lmodern}
\usetheme{metropolis}
 
\title{Determining Auto Mobile Insurance Risk}
\author[Ben Willis]{Ben Willis \\ \scriptsize{ Supervised by: Dr Matthias Troffaes}}
\institute{Durham University}
\date{2017}
 
\begin{document}
 
\frame{\titlepage}

\begin{frame}
	\frametitle{Problem}
	\begin{center}
		\begin{tabular}{l c c c c|c}
			Make       & Length & $\dots$ & Horsepower & Price & Risk \\
			\hline
			Volvo      & 188.8  & $\dots$ & 114        & 12940 & -2   \\
			Audi       & 192.7  & $\dots$ & 110        & 18920 & 2    \\
			Mitsubishi & 172.4  & $\dots$ & 116        & 9279  & 1    \\
			Audi       & 176.6  & $\dots$ & 115        & 17450 & ?
		\end{tabular}
	\end{center}
	\begin{itemize}
		\item Problem of determining the risk to an insurer of an auto mobile.
		\item Data set contains technical information about 205 vehicles and an expert's assessment of their risk.
	\end{itemize}
\end{frame}

\begin{frame}
	\frametitle{Naive Bayes Classifier}
		\begin{block}{Probabilistic Interpretation}
			Bayes theorem and the naivety assumption gives us:
			\begin{equation}
				P(c \mid \mathbf{a}) \propto P(c)\prod_{i=1}^{k}P(a_i \mid c)
			\end{equation}
		\end{block}
		\begin{block}{Choosing the Risk}
			\begin{equation}
				\hat c = \arg\max_{c \in \mathcal{C}} P(c)\prod_{i=1}^{k}P(a_i \mid c)
			\end{equation}
			This is known as the maximum a posteriori (MAP) estimate.
		\end{block}
\end{frame}

\begin{frame}
	\frametitle{Estimating Chances}
	To estimate these probabilities we parametrise them.
	\begin{block}{The Likelihood Function}
		\begin{equation}\label{likelihood function}
			l(\mathbf{\theta} \mid \mathbf{n}) \propto \prod_{c \in \mathcal{C}} \left[ \theta_c^{n(c)} \prod_{i=1}^k \prod_{a_i \in \mathcal{A}_i} \theta_{a_i \mid c}^{n(a_i, c)} \right]
		\end{equation}
		\begin{description}
			\item[$\theta_c, \theta_{a_i \mid c}$] Parameter for unknown $P(c), P(a_i \mid c)$
			\item[$n(c), n(a_i, c)$] Number of times observed
		\end{description}
	\end{block}
	\begin{block}{Maximum Likelihood Estimates}
		\begin{equation}
			\hat{\theta}_c = \frac{n(c)}{N} \qquad \hat{\theta}_{a_i \mid c} = \frac{n(a_i, c)}{n(c)}
		\end{equation}
	\end{block}
\end{frame}

\begin{frame}
	\frametitle{Application and Results}
	\begin{itemize}
		\item Discretize continuous variables
		\item Discard objects with missing values
		\item Estimate probabilities using maximum likelihood estimates
		\item Assign class using MAP estimate
		\item Accuracy of 59.95\% using 10-fold cross validation
	\end{itemize}
\end{frame}

\begin{frame}
	\frametitle{Introducing the Prior}
	\begin{block}{Prior Distribution}
		\begin{equation}
			f(\mathbf{\theta} \mid \mathbf{t}, s) \propto \prod_{c \in \mathcal{C}} \left[ \theta_c^{st(c) - 1} \prod_{i=1}^k \prod_{a_i \in \mathcal{A}_i} \theta_{a_i \mid c}^{st(c, a_i) - 1} \right]
		\end{equation}
		Hyper parameters $s>0$ and $\mathbf{t}$ such that:
		\begin{equation}
			\sum_{c \in \mathcal{C}} t(c) = 1 \qquad \sum_{a_i \in \mathcal{A}_i} t(a_i \mid c) = t(c) \qquad t(a_i \mid c) > 0
		\end{equation}
	\end{block}
	For uninformative prior set:
	\begin{equation}
		s = 1 \qquad t(c) = \frac{1}{|C|} \qquad t(a_i, c) = \frac{1}{|A_i||C|}
	\end{equation}
\end{frame}

\begin{frame}
	\frametitle{New Probability Estimates}
	\begin{block}{Posterior Expectations}
		\begin{align}
			E(\theta_c \mid \mathbf{n},s,\mathbf{t}) & = \frac{n(c) + st(c)}{N + s} = \hat{\theta}_c \\ E(\theta_{a_i \mid c} \mid \mathbf{n},s,\mathbf{t}) & = \frac{n(c) + st(a_i, c)}{N + st(c)} = \hat{\theta}_{a_i \mid c}
		\end{align}
	\end{block}
\end{frame}

\begin{frame}
	\frametitle{Reapplication}
	\begin{itemize}
		\item Measure Mean Squared Error (MSE)
		\item Also measure random classifications
	\end{itemize}
	\begin{tabular}{ l|c c c }
		              & Accuracy & MSE   & Random Classifications \\
		\hline
		Maximum Likelihood    & 59.95\%  & 2.823 & 22.45\%        \\
		Posterior Expectation & 68.17\%  & 0.689 & 0\%
	\end{tabular}
\end{frame}

\begin{frame}
	\frametitle{Future Work}
	\begin{itemize}
		\item Investigate changes in hyper parameters
		\item Test alternate decision methods
		\item Make use of vehicles with missing values
	\end{itemize}
\end{frame}
 
\end{document}