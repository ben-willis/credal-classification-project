\documentclass{beamer}
 
\usepackage[utf8]{inputenc}
\usepackage{lmodern}
\usetheme{metropolis}
 
\title{Determining Automobile Insurance Risk}
\author[Ben Willis]{Ben Willis \\ \scriptsize{ Supervised by: Dr Matthias Troffaes}}
\institute{Durham University}
\date{2017}
 
\begin{document}
 
\frame{\titlepage}

\begin{frame}
	\frametitle{Problem}
	\begin{center}
		\begin{tabular}{l c c c c|c}
			Make       & Length & $\dots$ & Horsepower & Price & Risk \\
			\hline
			Volvo      & 188.8  & $\dots$ & 114        & 12940 & -2   \\
			Audi       & 192.7  & $\dots$ & 110        & 18920 & 2    \\
			Mitsubishi & 172.4  & $\dots$ & 116        & 9279  & 1    \\
			Audi       & 176.6  & $\dots$ & 115        & 17450 & ?
		\end{tabular}
	\end{center}
	\begin{itemize}
		\item Problem of assessing the risk to an insurer of an automobile
		\item Data set contains technical information about 205 vehicles and an expert's assessment of their risk
		\item Risk rating on integer scale from -2 to 3
	\end{itemize}
\end{frame}

\begin{frame}
	\frametitle{Naive Bayes Classifier}
	\begin{block}{Notation}
		\begin{description}
			\item[$c$] Observed Class (Risk Rating)
			\item[$\mathbf{a} = (a_1,...,a_k)$] Observed Attributes
		\end{description}
	\end{block}
	\begin{block}{Probabilistic Interpretation}
		Bayes theorem and the naivety assumption gives us:
		\begin{equation}
			P(c \mid \mathbf{a}) \propto P(c)P(\mathbf{a} \mid c) = P(c)\prod_{i=1}^{k}P(a_i \mid c)
		\end{equation}
	\end{block}
\end{frame}

\begin{frame}
	\frametitle{Classifying an Object}
	Choose class which minimizes expected loss
	\begin{block}{0-1 Loss}
		\begin{equation}
			\hat{c} = \arg\max_{c \in \mathcal{C}} P(c \mid \mathbf{a}) = \arg\max_{c \in \mathcal{C}} P(c)\prod_{i=1}^{k}P(a_i \mid c)
		\end{equation}
		The maximum a posteriori (MAP) estimate.
	\end{block}
	\begin{block}{Quadratic Loss}
		\begin{equation}
			\hat{c} = E(C \mid \mathbf{a}) = \sum_{c \in \mathcal{C}} \left[ c \cdot P(c)\prod_{i=1}^{k}P(a_i \mid c) \right]
		\end{equation}
		The minimum mean square error (MMSE) estimate.
	\end{block}
\end{frame}

\begin{frame}
	\frametitle{Estimating Probabilities}
	Parametrise the probabilities:
	\begin{description}
		\item[$\theta_c$] Chance of a vehicle having risk rating $c$
		\item[$\theta_{a_i \mid c}$] Chance of a vehicle having attribute $a_i$ given that it has risk rating $c$
	\end{description}\vspace{0.5em}

	Denote the observed frequencies:
	\begin{description}
		\item[$n(c)$] Number of vehicles with risk rating $c$
		\item[$n(a_i, c)$] Number of vehicles with attribute $a_i$ and risk rating $c$
	\end{description}
	Note that $\sum_{c \in \mathcal{C}}n(c) = N$ and $\sum_{a_i \in \mathcal{A}_i}n(a_i, c) = n(c)$.
\end{frame}

\begin{frame}
	\frametitle{The Likelihood Function}
	\begin{block}{The Likelihood Function}
		\begin{equation}\label{likelihood function}
			l(\mathbf{\theta} \mid \mathbf{n}) \propto \prod_{c \in \mathcal{C}} \left[ \theta_c^{n(c)} \prod_{i=1}^k \prod_{a_i \in \mathcal{A}_i} \theta_{a_i \mid c}^{n(a_i, c)} \right]
		\end{equation}
	\end{block}
\end{frame}

\begin{frame}
	\frametitle{A Prior Distribution}
	\begin{block}{Prior Distribution}
		\begin{equation}
			f(\mathbf{\theta} \mid \mathbf{t}, s) \propto \prod_{c \in \mathcal{C}} \left[ \theta_c^{st(c) - 1} \prod_{i=1}^k \prod_{a_i \in \mathcal{A}_i} \theta_{a_i \mid c}^{st(a_i, c) - 1} \right]
		\end{equation}
		Hyperparameters $s>0$ and $\mathbf{t}$ such that:
		\begin{equation}
			\sum_{c \in \mathcal{C}} t(c) = 1 \qquad \sum_{a_i \in \mathcal{A}_i} t(a_i, c) = t(c) \qquad t(a_i, c) > 0
		\end{equation}
	\end{block}
	Let $s=1$ and for indifference:
	\begin{equation}
		t(c) = \frac{1}{|\mathcal{C}|} \qquad t(a_i, c) = \frac{1}{|\mathcal{A}_i||\mathcal{C}|}
	\end{equation}
\end{frame}

\begin{frame}
	\frametitle{Estimating Probabilities}
	\begin{block}{Maximum Likelihood Estimate}
		\begin{equation}
			\hat{\theta}_c = \frac{n(c)}{N} \qquad \hat{\theta}_{a_i \mid c} = \frac{n(a_i, c)}{n(c)}
		\end{equation}
	\end{block}
	\begin{block}{Posterior Expectation}
		\begin{equation}
			E(\theta_c \mid \mathbf{n},s,\mathbf{t}) = \frac{n(c) + st(c)}{N + s} \qquad E(\theta_{a_i \mid c} \mid \mathbf{n},s,\mathbf{t}) = \frac{n(a_i, c) + st(a_i, c)}{n(c) + st(c)}
		\end{equation}
	\end{block}
\end{frame}

\begin{frame}
	\frametitle{Application}
	\begin{itemize}
		\item Discretize continuous variables
		\item Discard vehicles with missing attributes
	\end{itemize}

	Measure three metrics:
	\begin{description}
		\item[Accuracy] Percentage of correct classifications
		\item[Random Assignments] Percentage of vehicles for which class randomly assigned ($P(c \mid \mathbf{a}) = 0$ for all $c$).
		\item[Mean Square Error] Average squared difference of the assigned class and true class
	\end{description}
\end{frame}

\begin{frame}
	\frametitle{Results}
	Use MAP estimate for class and vary probability estimates:
	\begin{center}
		\begin{tabular}{ l|c c }
			                      & Accuracy & Random Assignments\\
			\hline
			Maximum Likelihood    & 59.95\%  & 22.45\% \\
			Posterior Expectation & 68.17\%  & 0\%
		\end{tabular}
	\end{center}
	\begin{itemize}
		\item Maximum likelihood estimates lead to less accurate classifications
		\item Use of prior means class never randomly assigned
	\end{itemize}
\end{frame}

\begin{frame}
	\frametitle{Results}
	Use posterior expectation for probability estimates and vary decision mechanism:
	\begin{center}
		\begin{tabular}{ l|c }
			              & Mean Square Error   \\
			\hline
			MAP Estimate  & 0.642 \\
			MMSE Estimate & 0.565
		\end{tabular}
	\end{center}
	\begin{itemize}
		\item MMSE estimate closer to true risk rating
		\item MMSE estimate may not be an integer
	\end{itemize}
\end{frame}

\begin{frame}
	\frametitle{Future Work}
	\begin{itemize}
		\item Investigate sensitivity to changes in hyperparameters
		\item Test alternate decision methods
		\item Make use of vehicles with missing values
	\end{itemize}
\end{frame}

\begin{frame}[standout]
	Questions?
\end{frame}

\begin{frame}[allowframebreaks]{References}

  \bibliography{demo}
  \bibliographystyle{abbrv}

\end{frame}
 
\end{document}